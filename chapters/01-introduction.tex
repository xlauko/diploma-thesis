\chapter{Introduction}\label{ch:Introduction}

%\begin{chapquote}{Monty Python}
%Nobody Expects the Spanish Inquisition.
%\end{chapquote}

\noindent

In everyday practice, software engineers struggle to achieve correctness of
software. Nowadays, a standard approach used by the industry is testing, which
enables developers to cover basic use cases of their programs. However, with the
increasing size of applications, it becomes harder to achieve satisfactory
coverage, and include all the corner cases.

Moreover, testing is not sufficient when it encounters nondeterminism in a
program. We may distinguish two types of nondeterminism. Data nondeterminism
which is introduced by inputs from the environment, and control-flow
nondeterminism caused by unpredictable interleaving of threads in parallel
programs.

To avoid the problems of testing, many techniques of formal verification have
been invented, such as model checking~\cite{Baier08}, abstract
interpretation~\cite{Cousot14}, symbolic execution~\cite{King76}. These
techniques try to check all the behaviours of the program automatically, and
hence cover all the corner cases.

The technique that we are particularly interested in is explicit-state model
checking.\sidenote{Explicit-state model checking is a technique, which
enumerates all possible states of a program and checks for the satisfaction
of a given property.} The primary use case for model checking is to
verify some property over the concurrent program by analysis of all
possible interleavings of its threads. This is a significant advantage
against testing, since by testing we are not able to predict all thread
interleavings and the tests may behave nondeterministically. Besides all
the advantages of model checking, it comes with a drawback, that it is
useless on programs that interact with an environment. In this case,
model checking has to verify executions for each possible input, what
requires an unmanageable amount of work -- an exploration of an enormous
state space (also known as a \emph{state space explosion
problem}~\cite{Clarke99}).

To tackle the problem with program inputs, a few extensions of explicit-state
model checking have been introduced over the time -- abstraction-based model
checking~\cite{Clarke94} and symbolic model checking~\cite{Clarke96}. These
techniques try to group inputs that do not induce changes in the behaviour of
the program, and verify programs only according to those groups (or a
representant is taken from the group). For example, consider a function
that checks whether a number is greater than zero: we can separate all the
possible inputs into two groups -- numbers bigger than zero, and the rest. Then
we need to verify the program only according to those groups (i.e. just for the
two possible runs).

In this thesis, we would like to introduce a new approach to deal with inputs.
In the approaches mentioned before, the nondeterminism brought by inputs is
commonly processed directly by the verification tool. Our idea is to transform
the program in such a way that it will simulate the nondeterminism of inputs
independently of the verification tool (i.e. the grouping of the inputs). Hence,
the verification tool does not need to provide support for input
handling (data nondeterminism). Moreover, explicit-state model
checkers should be able to verify such a program without suffering
from state space explosion.

To be precise, we want to create a tool that will take a program and replace the
nondeterministic data by some arbitrary representation of the group of inputs
(i.e.~some abstraction of the input).  Likewise, we want to transform the
operations on the input data to the operations on the groups of
data.\sidenote{For example, when we subtract a value one from the group of
values $\{1,5,12\}$ the~resulting group is $\{0,4,11\}$.} Traditionally, a
similar effect is achieved by interpreting the program and representing the
nondeterministic data internally in the verification tool. Our proposal is
not to complicate the verification tool, letting the program do the work and
manipulate abstract values directly. To achieve this, we suggest that
instead of interpreting the program instructions in an abstract way, the
abstraction can be compiled into the program directly (using a program
transformation). Hence, the transformed program will manipulate with
an abstract representation of the input instead of a concrete value.  The
program transformed in this way, we can forward for analysis to an arbitrary
verification tool.

In this work, we will present how such a transformation can be achieved and how
abstract data can be represented. The structure of the thesis is following:
first in \autoref{ch:preliminaries}, we will cover the input language for our
transformation (\LLVM), and we will define what is model checking more
precisely. In \autoref{ch:divine}, we will look on \DIVINE model checker
toolchain \cite{Divine17}, in which is the program transformation implemented.
In this chapter, we will also cover other approaches to the input abstraction,
from which we have taken inspiration. More specifically, the \SymDIVINE
approach of handling inputs will be explored \cite{Mrazek16}.
\autoref{ch:abstraction} presents basic abstractions and the whole process of
program transformation. The experimental evaluation of the work is summarised
in the \autoref{ch:results}. Evaluation has been made on the subset of
\textsc{sv-comp} (software verification competition) benchmarks
\cite{Beyer17} and on the set of handcrafted data structures, such as an AVL
tree. Finally, \autoref{ch:conclusion} summarizes contributions of the work.

