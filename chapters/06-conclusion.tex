\chapter{Conclusion}\label{ch:conclusion}

In this thesis, we have explored how to deal with inputs in the program
verification. A common approach is to abstract inputs and do a kind of abstract
interpretation of a program. However, to add this feature into a verification
tool, in many cases requires to tweak the tool a lot.

Therefore, we have proposed a novel approach to abstraction, as program
transformation. In this way we are able to abstract a program with a minor
modification to the verification tool.

Specifically, we have focused on a symbolic abstraction as used in the \SMT-based
symbolic model checkers. We have recognized which parts of the program can be
abstracted and what modification of the tool needs to be done.

The entire approach was designed as transformation of \LLVM programs for
explicit-state model checker \DIVINE. But we believe, that it can be easily extended
to another tools.

\section{Contribution}
We believe that this thesis contributed to the verification world, by enabling
easier abstraction of programs, which will not be able without a large amount of
programing work.

First and foremost, we would like to point out that the largest part of this
thesis is contained in the implementation. The sole transformation required
a lot of static analysis as was described in \autoref{ch:abstraction}. The main
implementation was done as extension of the tool \LART, where all analyses can be
found.

The transformation was evaluated with a symbolic domain\sidenote{Implementation
of a symbolic domain and an algorithm in \DIVINE is not a part of this thesis.},
what showed that this approach can compete with traditional symbolic model
checking.  At the same time we have achieved that a new abstraction can be
easily implemented and used in already settled \DIVINE toolchain. As a proof
we have created a simple domain that distinguishes whether the value is zero
or not.

Additionally we provide a short tutorial how to abstract a program and verify it by
\DIVINE, in the \autoref{ch:appendixb}. A short description can be also found on
the project website \url{divine.fi.muni.cz} in the manual section.

\section{Future work}

The sole abstraction framework opens a broad area in the abstraction of
programs. An example is a predicate abstraction, which can be done in a
similar manner as a symbolic one.

This brings us to the state of the art approaches such as \textsc{cegar} loop.
To refine an abstraction, our framework should be easily extended with other
analyses. The main mission here is to find a way how to extract an counter
example from the abstracted program. Further, the transformation should be done
according to the extracted counter example.

In our framework we have covered only a subset of \LLVM. In this way we see a
multiple directions in which we can extend the current framework. Currently, the
biggest limitation is abstraction of memory and pointers. To do their
abstraction, the framework needs to be extended with some pointer analysis.
Current implementation is also restricted in the abstraction of aggregate types,
where we are not able to abstract array types, and completely the struct
types. In this case for complete abstraction a shape analysis is
required. Lastly, the transformation can be easily extended to abstract
floating point operations similarly as integers.

For other domains than symbolic, a full constraint propagation needs to be
implemented. To do so, we need to find a compromise between analysis and the
constraint computation during the runtime.

Lastly as in every software, a lot of potential optimization can be done in the whole approach.
Meaning optimizations of the sole transformation and also the program which is
abstracted. For example, one optimization mentioned before is an inlinig of abstract
operations.

And as promising this tool may sound, we aim to use it with other tools too.

