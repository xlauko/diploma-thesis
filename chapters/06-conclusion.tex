\chapter{Conclusion}\label{ch:conclusion}

In this thesis, we have explored how to deal with inputs in program
verification. A common approach is to abstract inputs and perform an
abstract-interpretation-like analysis of the program. However, adding this feature into a
verification tool requires significantly tweaking the tool in many cases.

Therefore, we have proposed a novel approach to verification of programs with
inputs. Specifically, an abstraction via program transformation. This way, we
were able to abstract a program with only minor modifications of the verification
tool.

Especially, we have focused on symbolic representation of inputs as used in \SMT-based
symbolic model checkers. To do so, we have identified which parts of the program can be
abstracted and what modifications of the verification tool need to be done.

The transformation was designed for \LLVM{} programs, and evaluated with
explicit-state model checker \DIVINE. But we believe that it can be easily extended
to other tools.

\section{Contribution}
We believe that this thesis contributed to the verification world by enabling
easier abstraction of programs, which would not have been possible without a
large amount of programming work. Moreover, we have explored an abstraction
of \LLVM bitcode and defined composable algorithms for bitcode transformation.

First and foremost, we would like to point out that the most significant part of
this thesis concerns the implementation. The transformation alone required
a lot of static analysis as was described in \autoref{ch:abstraction}. The
implementation was done as an extension of the tool \LART, where all building
blocks of the transformation are located.

The transformation was evaluated with a symbolic domain,\marginpar{Implementation
of the symbolic domain and a symbolic exploration algorithm in \DIVINE is not part of this thesis.}
which showed that this approach can compete with traditional symbolic model
checking. At the same time, a new abstractions can be easily implemented and
integrated into the mature \DIVINE toolchain. To demonstrate this, we have created a
simple domain that distinguishes whether a value is zero or not.

Additionally, in \autoref{ch:appendixb} we provide a short tutorial on how to
abstract a program and verify it with \DIVINE. A description is also
available from the manual, on the project website -- \url{divine.fi.muni.cz}.

\section{Future work}

The proposed framework brings a number of new possibilities in abstraction of
programs. An example is predicate abstraction~\cite{Flanagan02}, which can be
implemented in a similar manner as the symbolic one.

A useful extension for the abstraction framework can also be an implementation
of a \textsc{cegar} loop~\cite{Clarke20}. In current settings, we are able to
abstract a program, verify it and find a potential counter-example. The
remaining challenge is how to extract from the counter-example informations
for refinement. The refinement can be then implemented as a more precise
transformation.

In our framework, we have covered only a subset of program behaviours.
Currently, the biggest limitation is abstraction of memory and pointers.
Abstraction of those, can be done either naively, where program would build
a lookup table for pointers and will simulate the pointers based
on the values in the table. This approach has a big overhead on the runtime
interpretation. A better direction should be to use a pointer analysis, to
restrict pointers during the transformation.

The current implementation is also restricted in the abstraction of aggregate
types, where we are not able to abstract array types, and there are limitations
in abstraction of struct types. In this case, we can also let the runtime do the
hard work and maintain the tables of abstract values, but for faster
verification we can extend the transformation with a shape analysis. Finally,
the transformation can be easily extended to abstract floating
point operations similarly to integers.For domains other than
symbolic, better constraint propagation can be implemented. To do
so, we need to find a compromise between analysis and the
constraint computation during the runtime.

Finally, as in every software, there is a lot of potential for optimization.
This includes optimizations of the transformation and also of the abstracted program.
An example of one such optimization is inlining of abstract operations. In this
way, we can improve the performance of verification tool.

Given the promising results (\autoref{ch:results}), we would like to combine
\LART with verification tools other than \DIVINE.
