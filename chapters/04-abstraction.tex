\chapter{Abstractions}\label{ch:abstraction}

In this chapter, we will propose a novel approach to tackle
handling of program inputs in the model checking. We aim to provide a tooling to
enable abstraction of inputs, with only minor modifications to model checker.

One of the approaches, as mentioned in \autoref{ch:divine}, is symbolic model checking.
From the point of abstractions, symbolic representation forms an exact abstraction
of program data. In this chapter, we will review how programs are abstracted more generally.
Thereafter we will present an approach to abstractions via transformations
in the preprocessing phase of model checking.

\section{Approaches to abstraction}

In common there are two main approaches how to do abstraction of programs.
Traditionally, abstraction engines work as interpreters~\cite{Cousot79}. During
interpretation, the abstraction tool needs to interpret given code according
to some semantics (either encoded in the tool or given). In similar manner works
\SymDIVINE. During the interpretation of \LLVM bitcode \SymDIVINE accordingly
manipulates formula representing the multi states, see \autoref{sub:symdivine}.

The other alternative to do abstraction is to compile the abstraction into the
program. It means that the program will be transformed in such manner that it will
manipulates with abstract values instead of concrete ones. For example, in the
case of symbolic abstraction, the abstracted program manipulates directly with
formulae, instead of concrete variables. The abstracted program after such
transformation behaves almost as a traditional program. The main difference is
that the abstracted program may have lost precision, hence it needs to do
in some cases nondeterministic choices.

When comparing the above mentioned approaches, each of them comes with its pros
and cons. The transformation approach does not complicate the design of the
verification tool since it is used before the very verification process. Hence
minimize the probability of bringing a bug into the verification tool itself.

Bonus of doing the verification over the transformed program is also that,
besides verifying the sole program, the verification tool checks the
correctness of the abstract transformers, since they are encoded into the
program. This is not possible in interpretation approach, because the
transformers are part of the very interpreter.

Additionally the transformation approach enables reuse of the transformation
with multiple verification tools. Also the program needs to be transformed only
once, and thereafter it can be reused, in compare to interpretation where the
analysis has to be done each time.

The both approaches enables pluggable abstract domains, by generalizing their
common analysis, as it will be shown in this chapter.

But on the other hand, the transformation approach requires more complicated
design, since the computation has to be done statically. Also the sole
verification of transformed program may potentially become slower, since the
abstract operations have to be interpreted by the verification tool, what may
require nontrivial time in compare to direct interpretation in abstract
interpreters.

As long as \DIVINE is strong foundation for the explicit model checking, we have
decided not to complicate the precisely crafted interpreter, and take the path of
the second approach. We expect the program transformation approach to generalize
the common analysis used for the abstraction, hence provide a toolset that will be
easily extended to other abstractions.

Besides having the symbolic abstraction as a main goal, during the implementation
we have realized that the core of transformation is usable for arbitrary
abstraction, as will be shown in this chapter.

This gives us a goal to determine the general technique how to create an arbitrary
abstraction of a program with minimal effort to be done. As reasonable we see
that, given a domain and set of transformers with a lowering function from
concrete to abstract transformers is sufficient description for our tool.

Since the transformation of this kind is not known to be done by any other
\LLVM based tool, we consider this as broadening of field knowledge and hope for
usable results.

\section{Abstract domains} \label{sec:absdom}

When dealing with abstraction of programs, first of all, one has to define an
abstract domain for program variables and transformers based on operational
semantics of a program. From model checking point of view we want to define an
abstraction function $\alpha \colon C \to A$ transforming a concrete state $C$
to an abstract state $A$, where $C$ corresponds to some memory valuation of some
tuple of concrete values and $A$ corresponds to tuple representing a concrete
state in abstract world, see \autoref{subsec:amc}. Additionally we have to
define corresponding abstract transformers of the states (in our case abstract
equivalents of \LLVM instructions). These transformers induces the transition
relation between abstract states. We kindly refer to \autoref{subsec:amc}.

\bigskip

\hrule

\bigskip
\noindent
\textit{An abstract domain is an abstract algebra, implemented as a library module,
providing a description of abstract program properties and abstract property
transformers describing the operational effect of program instructions and
commands in the abstract program. Abstract domains are often complete lattices. --- Cousot \cite{Cousot79} }
\bigskip

\hrule

\bigskip

To define a concrete domain over \LLVM we have to represent integer types,
pointer types and aggregate types. For simplicity we will work only with this
subset of \LLVM types. Moreover we do not permit abstraction of pointers, since
pointer analysis is beyond the scope of this thesis, but we will permit
pointers to abstract values.

Generally we define integer types as set $C_{i}$ containg all \LLVM integer
types (i.e.~a type for each bitwidth).
Aggregate types $C_a$ are then defined as all finite products of integer types and
pointer types, and set of pointer types $C_p$ consists of pointers to all integer
and aggregate types. The whole type system of concrete domain is
then defined as union of these sets: $C = C_i \cup C_a \cup C_p$. We will denote
set of all the possible concrete values by $V_C$.

Defining an abstract domain $A$ over the \LLVM type system requires a representation of
abstract variables, definition of the type transformation function $\alpha_A \colon C \to A$
and a function for values abstraction $\textit{lift} \colon \mathcal{P}(V_C) \to
V_A$, where $\mathcal{P}(V_C)$ denotes a potential set of $V_C$.
Similarly the transformation from abstract domain is defined as
$\textit{lower} \colon V_A \to \mathcal{P}(V_C)$. We remark that elements of $V_C$
are sets of possible values.

%\marginpar{A well defined abstract domain traditionally forms a complete lattice.
%Hence the operations over the abstract values may be defined in the notion of
%lattice operations (meet and join).
%
%\bigskip
%\resizebox{\marginparwidth}{!}{
%\begin{tikzpicture}
%    \node [label = \textit{unknown}] (u) {\textbullet};
%    \node [label ={165:\textit{zero}}, below left = 0.5 cm of u ] (z) {\textbullet};
%    \node [label ={15:\textit{nonzero}}, below right = 0.5 cm of u] (n) {\textbullet};
%    \node [below = 1.1 cm of u] (b) {$\bot$};
%    \draw (u) -- (z);
%    \draw (z) -- (b);
%    \draw (u) -- (n);
%    \draw (n) -- (b);
%\end{tikzpicture}
%}
%
%A complete lattice of $Z$ domain contains also a $\bot$ element. Which
%represents a value in unspecified state. This state will be not reachable in
%our $Z$ domain so we will not consider it in domain implementation.
%\label{fig:lattice}}

Since we do not permit abstract pointers, the $\alpha$ function for pointer types and will be
always defined as transformation from concrete pointer type to abstract pointer
type with corresponding abstracted base type. Similarly the transformation of
aggregate types may be done in per element manner. Hence only the definition of
integer types transformation is interesting. Similar reasoning applies to
\textit{lift} and \textit{lower} functions, where only transformation of the
integer values is needed to by specified.

As example of simple domain we may define a three value domain $Z$ where only
one type exists with following possible values:
\[ V_{Z} = \{ \textit{zero}, \textit{nonzero}, \textit{unknown} \}\]


The values represent whether the abstracted variable is zero, non-zero or that we do
not know the precise value of the variable. Since we have only one type
$\alpha_{Z}$ maps all integer types to single possible type.
The definition of $\textit{lift}$ and $\textit{lower}$ for integer
types in $Z$ domain is straightforward:

\[
  lift(c) =
  \begin{cases}
    \textit{zero}    & \text{if } \{0\} = c, \\
    \textit{nonzero} & \text{if } 0 \not \in c, \\
    \textit{unknown} & \text{else.}
  \end{cases}
\]

\noindent
Because lower depends on resulting concrete integer type $\texttt{i}k$ we define separate
function for each bitwidth.

\[
  lower_{\texttt{i}k}(a) =
  \begin{cases}
    \{0\}      & \text{if } a = \textit{zero},\\
    \{1, 2^k\} & \text{if } a = \textit{nonzero}, \\
    \{0, 2^k\} & \text{else}. \\
  \end{cases}
\]

Our main goal is to allow the domain to be specified in some common higher-level
language and consequently compiled into the verified program. For this purpose,
we have chosen \Cpp{} language. To represent the $Z$ domain in
\Cpp{} we have created a simple enumeration type:
\begin{minted}{cpp}
struct Zero {
    enum Domain { ZeroValue, NonzeroValue, Unknown }
    Domain value;
};
\end{minted}
In this way we are able to leverage the existing type system of \LLVM and
using the compilation encode an abstract values as \LLVM scalars.

Additionally to domain specification a corresponding transformers have to be
defined. In concrete domain the set of transformers is formed by \LLVM
instructions. Concretely binary arithmetic operations, bitwise binary
operations, memory accesses, conversion operations and comparison instructions.

For abstract domains we have to define equivalent operations over the abstract
values. Implementation of the operations may be provided as \Cpp{} functions over
the defined abstract types. For example an abstract equivalent of \texttt{add}
instruction in $Z$ domain may be implemented as follows. \footnote{The whole
\texttt{zero} domain implementation can be found in thesis attachments, with
only minor technical differences.}
\begin{minted}{cpp}
Zero __abstract_zero_add( Zero a, Zero b ) {
    if ( a.value == Zero::Domain::ZeroValue )
        return b.value;
    if ( b.value == Zero::Domain::ZeroValue )
        return a.value;
    return Zero::Domain::Unknown;
}
\end{minted}

Generally the binary bitwise, arithmetic and comparison operations are of
type $Z \times Z \to Z$. Moreover we define cast operations of type $Z \to Z$,
and operations for memory accesses \code{alloca}, \code{store} and \code{load}. The abstract \code{alloca} instruction allocates a new \code{Zero}
variable and returns a pointer to it. The \code{load} and \code{store} access memory as expected.

\subsection{Domain interactions} \label{sec:interactions}
In cases when different concrete domains interact, a special care needs to be
taken. We will concern about two major types of interactions. First one occurs
with casting operations (bit-casting, width extension, truncation, etc.), when
values are transformed from one domain to another. The second
interaction between domains occurs during comparisons, where \LLVM produces from
some predicate over various values a boolean result. On the boolean result
usually depends some branching, which needs to take concrete value. Hence the
abstract result from comparison needs to produce a concrete value. But since the
result of the comparison may be uncertain, we need to do a nondeterministic
choice to produce either \emph{true} or \empf{false}.

To unify this interaction with control flow between domains, we introduce
a tristate domain (\emph{true, false, maybe}),
and enforce all abstract domains to define a transformer from their abstract
boolean to tristate domain. We will refer to it as \texttt{bool\_to\_tristate}
operation. Hence before each branching on the abstract value we may transform it
to generalized form of tristate and according to value of tristate decide to do
the nondeterministic choice. The abstract control flow will be further described
in \autoref{subsec:cf}.

Handling of cast interactions is much more tricky, generally a combinatorial
amount of code is required for transformation between abstract domains. The
simplest way is to treat bit-casts as a way of aliasing and enforce both
variables to be abstracted into the same abstract domain \cite{Rockai15}.


\subsection{Control flow of abstract programs} \label{subsec:cf}
When a branch is taken due to a \emph{maybe} result of some boolean expression,
the effect of the branch can be, in addition to its usual concrete semantics,
to restrict the values of some abstracted variables. Abstract interpreters
naturally take advantage of this fact, by building up a path condition
\cite{Rockai15}.

In transformation based approach we can take the advantage of a same principle, by
restricting values on which the branch has been taken. For this purpose we
facilitate an \code{assume} operation provided by abstract domain. The
\code{assume} operation has to be able to compute a new value for constrained
variable from given \LLVM predicate, such as \code{icmp}, and actual result of
that predicate (i.e.~\emph{true} or \emph{false}). Control flow of abstract
programs is further examined in \autoref{sec:bcp}.

\section{Symbolic domain}\label{sec:sym}

In compare to simple domain $Z$, symbolic domain brings a few complications.
First of all appears a question, how to define a symbolic data, and consequently
implement manipulations over them. In \SymDIVINE algorithm, the data were
described by two parts \emph{data definitions} and \emph{path condition} (see
\autoref{sub:symdivine}). We would like to bring this notion of separation of data
and path condition also to the abstracted program.

From already settled abstraction framework we may conclude that abstract
transformers corresponding to \LLVM instructions will facilitate on data
definitions, i.e.~the symbolic definitions are creted either via symbolic
\code{alloca} or as results of symbolic instructions. The path condition should
reflect the already taken path in the program, hence it will be described by
assumes that were encountered during the interpretation (i.e.~constraints given
by branching).

Since symbolic data are meant to be represented as bit-vector formulae,
a common sense brings us to solution, where abstract data will carry their
definitions in the way of some representation of formulae. A common formula
representation are formula trees. For example having the following program the
data representation at the end of program by formula trees will be as on picture bellow.
\begin{minted}{cpp}
int x = input();
int y = input();
int a = x + y;
int b = a * x;
\end{minted}

\begin{center}
\begin{minipage}{.4\textwidth}
\begin{tikzpicture}[->,>=stealth',level/.style={sibling distance = 2cm/#1, level
    distance = 1cm}]
    \node [ftreenode, label={a}] {$+$}
        child{ node [ftreenode] {$x$} }
        child{ node [ftreenode] {$y$} }
    ;
\end{tikzpicture}
\end{minipage}
\begin{minipage}{.4\textwidth}
\begin{tikzpicture}[->,>=stealth',level/.style={sibling distance = 2cm/#1, level
    distance = 1cm}]
    \node [ftreenode, label={b}] {$*$}
        child{ node [ftreenode] {$a$} }
        child{ node [ftreenode] {$x$} }
    ;
\end{tikzpicture}
\end{minipage}
\end{center}

You may have noticed that the image lacks the representation of $x$ and $y$
variables, it is because there are no restrictions on input variables so we will
consider them as free variables in formula. Another observation is that the tree
roots correspond to values produced by \SSA form of the program.

These trees are then easily transformed to formulae and given \SMT solver during
the run of symbolic exploration. The further details on creation of path
condition and symbolic algorithm will be described in \autoref{sec:symbolic}.

The transformers in symbolic domain can be then easily defined as operations on
the tree representation (i.e.~each register returning operation creates a new
tree representing the operation with arguments as leafs).

\section{Abstraction via program transformation}

In following sections we will describe the actual implementation of domain
injection into the program. By providing the common set of analysis we aim for
much easier implementation of new abstract domains. Furthermore we want to
create the transformation in such manner, that the resulting program will be a
runnable binary, provided the nondeterministic choice implementation (random
value or bit-vector of choices). Hence the abstracted program should be
analyzable by arbitrary tool, not just by \DIVINE. Moreover to preserve
the soundness of further analysis we have to define the transformation in
information lossless manner.

Ergo, we have chosen the transformation to be done on the \LLVM bit-code level to
mimic the logic of original program as precisely as possible. The whole
transformation is done in successive standalone \LLVM passes, producing the
abstracted \LLVM bitcode. Further described in following sections, and put
together in \autoref{fig:transformation}.

\begin{figure}[!ht]

\resizebox{\textwidth}{!}{
\begin{tikzpicture}[>=stealth',shorten >=1pt,auto,node distance=4em, <->]
\tikzset{>=latex}

\tikzstyle{every node}=[align=center, minimum width=1.25cm, minimum height=0.6cm]
\tikzset{empty/.style = {minimum width=0cm,minimum height=1cm, text width = 2
    cm, color = pruss}}
\tikzset{pass/.style = {
    draw,
    inner sep=3pt,
    rounded corners= 3pt,
    minimum width = 0cm,
    minimum height = 1.5cm,
    text width = 3cm,
    node distance = 4cm,
    text centered,
    color=pruss,
    fill=pruss!10
}}

\node [pass] (VPA) {Value propagation};
\node [pass, right of = VPA] (AI) {Abstraction of instructions};
\node [pass, right of = AI] (BCP) {Constraint propagation};
\node [pass, fill=vivid!40, right of = BCP] (DI) {Domain insertion};

    \node [empty, left = 0.5 cm of VPA]  (in) {bitcode \\ + \\ domains};
\node [empty, right = 0.5 cm of DI]  (out) {abstracted bitcode};

\draw [flow] (in) -- (VPA);
\draw [flow] (VPA) -- (AI);
\draw [flow] (AI) -- (BCP);
\draw [flow] (BCP) -- (DI);
\draw [flow] (DI) -- (out);
\end{tikzpicture}
}

\caption{Abstraction of program is done in four major steps. Firstly the
analysis of abstract values propagation through program is done. After that,
abstraction of instructions into intermediate \LART instructions is done.
Thereafter the \code{assume} calls are inserted to program and the given
constraints are propagated through bitcode. And finally the intermediate
abstract representation is replaced by appropriate operations in given domain.
We found that first three analysis are domain independent and can be done
without providing the actual domain implementation.
}
\label{fig:transformation}
\end{figure}


Providing the program and used domains implementations, one has
to annotate which variables should be abstracted in which domain. As in
following example the variable \code{x} is annotated as symbolic.
\begin{minted}{cpp}
int foo(int a, int b) {
    _SYM int x;
    int y = 0;
    if ( x < a ) {
        y = x + 4;
    } else {
        y = bar(a, b);
    }
    y = bar(a, y);
    return y;
}
\end{minted}
Through the following exploration of program abstraction we will work with this
simple \Cpp{} program. For readability we will not expose the corresponding \LLVM
bitcode and will work schematically on control flow diagrams, on which the
mapping to actual \LLVM should be easily done. But for curious reader we attach the
concrete \LLVM transformation results in the Appendix.
\add{ add margin note about annotations in C }

\add{ Add llvm bitcode to appendix }

The corresponding control flow graph for example program is displayed in
\autoref{fig:exampleprogram}.

Having the annotated program we need to examine all the reachable instructions
from annotated variables. Further we will call them \emph{roots} of abstraction.
The examination of the reachable instructions is done by \emph{value propagation
analysis} (VPA).
In future this part can be extend to alias analysis as described in
\cite{Rockai15}.

In order to generalize the further process and enable additional analysis we
postpone the abstract domain insertion until the whole program is analyzed and
transformed to suitable form.

Hence in the second step after VPA we introduce an intermediate representation of
abstract instructions. These instructions do not provide any implementation and
are represented in similar manner as \LLVM intrinsic functions
\cite{LLVM:langref}. During this phase (abstraction of instructions) we
\emph{lift} the reached instructions provided by VPA into the abstract form. The
only purpose of this intrinsics is to carry the domain information for further
analysis and eliminate analysis of domain implementation (since there is no need
to look into implementation of domains operations). Additionally the \LLVM
toolset can be used in this step for type checking of the applied transition.

In following analysis the \code{assume} calls are inserted and corresponding
backwards constraint propagation is executed. After all these analysis the
intermediate representation of abstract operations is replaced by proper domain
implementation.

\begin{figure}[!ht]
\begin{centering}
\resizebox{\textwidth}{!}{
\begin{tikzpicture}[>=stealth',shorten >=1pt,auto,node distance=4em, <->]
\tikzset{>=latex}
\tikzset{empty/.style = {minimum width=0cm,minimum height=1cm}}
    \node [llvm] (declx) {\_SYM i32 x };
    \node [llvm, below = 0.5 cm of declx] (decly) { i32 y};
    \node [br, below = 0.5 cm of decly] (br) { x < a };
    \node [llvm, below left = 0.5 cm of br] (brtrue) { y = x + 4 };
    \node [llvm, below right = 0.5 cm of br] (brfalse) { y = bar(a, b) };
    \node [llvm, below = 1.5 cm of br] (retcall) { y = bar(a, y) };
    \node [llvm, below = 0.5 cm of retcall] (ret) { ret y };

    \begin{pgfonlayer}{background}
        \node [bb, fit = (declx) (decly)] (bb1) {};
        \node [bb, fit = (brtrue)] (bb2) {};
        \node [bb, fit = (brfalse)] (bb3) {};
        \node [bb, fit = (ret) (retcall)] (bb4) {};
        \node [fun, label={ i32 foo(i32 a, i32 b)}, fit = (bb1) (bb2) (bb3) (bb4)] (fn) {};
    \end{pgfonlayer}

    \draw [flow, very thick] (decly) -- (br);
    \draw [flow, very thick] (br.west) -| (brtrue.north) node [near start, above = 5pt] {true};
    \draw [flow, very thick] (br.east) -| (brfalse.north) node [near start, above = 5 pt] {false};
    \draw [flow, very thick] (brtrue.south) |- (retcall.west);
    \draw [flow, very thick] (brfalse.south) |- (retcall.east);
\end{tikzpicture}
}
\end{centering}
    \caption{Control flow graph of example program. The gray
    boxes corresponds to \LLVM basic blocks and yellow diamond to conditional branching
    and corresponding boolean expression.}

    \label{fig:exampleprogram}
\end{figure}

\subsection{Value propagation analysis}
First of all, the analysis of reachable instructions have to be done. This
involves the analysis through calls and simple alias analysis for dealing with
pointers, output parameters and aggregate types. As a result of VPA a list of
functions with corresponding abstract roots is created.

\begin{example}\label{ex:vpa}
When analyzing the \code{foo} function, we compute the reachable set of
instructions using DFS from root \code{x} through all uses
\cite{LLVM:langref}. We investigate all instructions where x appears on the
left hand side (marked green on the left picture of \autoref{fig:vpa}).
Since the analysis found \code{store} (i.e.~\code{y = x + 4}) to variable
\code{y}, \code{y} is added to set of roots of the function \code{foo} and
marked with same domain as \code{x} is. Further the analysis is repeated
from the new root.

During the propagation analysis of \code{y} we found that \code{y} is an
argument \code{bar} function, hence we will need to analyze the \code{bar}
function and compute the roots for it. Note that even though the \code{bar}
function in the bottom basic block is called with an abstract argument we want
the call to \code{bar} in the \code{else} branch to remain explicit.

Another reached instruction worth mentioning is \code{ret}, since it means that
we will need to change the signature of \code{foo} function in further
instruction abstraction phase.

Since we did not found any \code{store} to some new local variable (marked on
the right of the \autoref{fig:vpa}) the analysis of \code{foo} function is
finished with set of roots $\{x, y\}$.

\begin{figure} [!ht]
\begin{center}
\resizebox{0.49\textwidth}{!}{
\begin{tikzpicture}[>=stealth',shorten >=1pt,auto,node distance=4em, <->]
\tikzset{>=latex}
\tikzset{empty/.style = {minimum width=0cm,minimum height=1cm}}
    \node [llvm, fill = apple!40] (declx) {\_SYM i32 x };
    \node [llvm, below = 0.5 cm of declx] (decly) {i32 y};
    \node [br, below = 0.5 cm of decly] (br) { x < a };
    \node [llvm, fill = apple!40, below left = 0.5 cm of br] (brtrue) { y = x + 4 };
    \node [llvm, below right = 0.5 cm of br] (brfalse) { y = bar(a, b) };
    \node [llvm, below = 1.5 cm of br] (retcall) { y = bar(a, y) };
    \node [llvm, below = 0.5 cm of retcall] (ret) { ret y };

    \begin{pgfonlayer}{background}
        \node [bb, fit = (declx) (decly)] (bb1) {};
        \node [bb, fit = (brtrue)] (bb2) {};
        \node [bb, fit = (brfalse)] (bb3) {};
        \node [bb, fit = (ret) (retcall)] (bb4) {};
        \node [fun, label={ i32 foo(i32 a, i32 b)}, fit = (bb1) (bb2) (bb3) (bb4)] (fn) {};
    \end{pgfonlayer}

    \draw [flow, very thick] (decly) -- (br);
    \draw [flow, very thick] (br.west) -| (brtrue.north) node [near start, above = 5pt] {true};
    \draw [flow, very thick] (br.east) -| (brfalse.north) node [near start, above = 5pt] {false};
    \draw [flow, very thick] (brtrue.south) |- (retcall.west);
    \draw [flow, very thick] (brfalse.south) |- (retcall.east);
\end{tikzpicture}
}
\resizebox{0.49\textwidth}{!}{
\begin{tikzpicture}[>=stealth',shorten >=1pt,auto,node distance=4em, <->]
\tikzset{>=latex}
\tikzset{empty/.style = {minimum width=0cm,minimum height=1cm}}
    \node [llvm, fill = apple!40] (declx) {\_SYM i32 x };
    \node [llvm, fill = apple!40, below = 0.5 cm of declx] (decly) { i32 y};
    \node [br, below = 0.5 cm of decly] (br) { x < a };
    \node [llvm, fill = apple!40, below left = 0.5 cm of br] (brtrue) { y = x + 4 };
    \node [llvm, below right = 0.5 cm of br] (brfalse) { y = bar(a, b) };
    \node [llvm, fill = apple!40, below = 1.5 cm of br] (retcall) { y = bar(a, y) };
    \node [llvm, fill = apple!40, below = 0.5 cm of retcall] (ret) { ret y };

    \begin{pgfonlayer}{background}
        \node [bb, fit = (declx) (decly)] (bb1) {};
        \node [bb, fit = (brtrue)] (bb2) {};
        \node [bb, fit = (brfalse)] (bb3) {};
        \node [bb, fit = (ret) (retcall)] (bb4) {};
        \node [fun, label={ i32 foo(i32 a, i32 b)}, fit = (bb1) (bb2) (bb3) (bb4)] (fn) {};
    \end{pgfonlayer}

    \draw [flow, very thick] (decly) -- (br);
    \draw [flow, very thick] (br.west) -| (brtrue.north) node [near start, above = 5pt] {true};
    \draw [flow, very thick] (br.east) -| (brfalse.north) node [near start, above = 5pt] {false};
    \draw [flow, very thick] (brtrue.south) |- (retcall.west);
    \draw [flow, very thick] (brfalse.south) |- (retcall.east);
\end{tikzpicture}
}
\end{center}
\caption{Value propagation analysis of \code{foo} function.}\label{fig:vpa}
\end{figure}
\end{example}

As described in \autoref{ex:vpa}, the value propagation analysis, besides
the analysis of reachable instructions, has to step into the function calls with
abstract arguments, marking the arguments as roots of the reached function.

Since the stepped-in function may have some annotated values we will distinguish
two types of roots, the annotation dependent and the argument dependent. Better
to show the meaning of distinction of roots on some example.

\begin{example} \label{ex:roots} Let's have following two functions.

\begin{minted}{cpp}
int bar(int arg) {
    _SYM int y;
    int b = arg;
    if (b < y)
        return b;
    else
        return 42;
}

int foo() {
    _SYM int x;
    int val = 0;
    int res1 = bar(x);
    int res2 = bar(val);
}
\end{minted}
    We can see that variable \code{x} is root for the function \code{foo} and
    \code{y} is the root for the function \code{bar}. Since \code{bar} is called
    with an abstract argument from \code{foo}, we want to add \code{arg} to
    roots of \code{bar}. But in further analysis we do not want to consider
    \code{bar} to be always called with an abstract value.

    Hence we distinguish two sets of roots for the \code{bar} function. One, when
    the function is called with a concrete value, including the single root
    $\{y\}$. For the case when the argument is abstract, the propagation
    results into a set $\{y, arg, b\}$. Since $\{y\}$ is always present in the
    set we represent the resulting roots as union of annotation dependent roots
    (i.e.~$y$) and argument dependent roots (i.e.~$arg, b$).
\end{example}

As described in the \autoref{ex:roots}, the result of value propagation analysis
is for each function computed a set of annotation dependent roots and sets of
argument dependent roots.

Additionally to the example, you may have noticed that depending on
argument, the result of the \code{bar} function can be abstract. To record this
information we also add the resulting value \code{res1} to annotation dependent
roots of the \code{foo} function, since the abstract call was made from the
propagation of the annotation roots.

Because we need to precisely determine the return type of function, we utilize
the \code{UnifyFunctionExitNodes} pass, provided by \LLVM toolset, to unify all
exit nodes to single one. Hence in the case
when the \code{bar} function returns abstract value, we need to \emph{lift} the
value \code{42}, since the returning of it is also possible.

Besides of propagation 'downwards' (in the direction of control flow), we need
to handle backwards propagation. This case occurs when some abstract value is
stored into the nonabstract output argument (pointer or reference) of the function,
illustrated below.

\begin{minted}{cpp}
void init(int *x) {
    _SYM int v;
    *x = v;
}

int main() {
    int x;
    init(&x);
}

\end{minted}

To handle the propagation backwards we mark the output argument as root of the
function, and examine all the locations from where the function is called. From
the call locations we compute the origin of the argument and mark it by the
domain that was stored to it (i.e. variable \code{x} is marked as symbolic root
in the \code{main} function).

Last of all we need to handle abstraction of aggregate types. Since we enable
only annotation of scalar types, abstract elements in aggregates may occur
only by storing to them. The problem to be solved is how to mark domins of
particular element of the aggregate. For this purpose we introduce tree-like
structure, that mimics the structure of aggregate type and stores domains of
its scalar elements (shown in \autoref{ex:fieldtrie}).

\begin{example} \label{ex:fieldtrie}
Let's have a nested structure represented in \LLVMIR:
\begin{minted}{llvm}
%Widget = type { i64, %Store* }
%Store = type { i64, i64, i64 }
\end{minted}
In \LLVM, when an aggregate is created by \code{alloca} instruction, we get
a pointer to the aggregate base. In order to access the elements we need to
\code{load} from the pointer to the aggregate. The concrete element is then
accessed by \code{getelementptr} instruction, which for given sequence of
indices computes an offset to the structure.

We have utilized this indexing strategy to create tree-like representation
of types. For example representation of \code{Widget} as tree is shown on the
left picture. The edges of the tree represent the access operation type
(i.e \code{load} operation or index in \code{getelementptr} instruction)
\cite{LLVM:langref}.

Since we work only with abstraction of scalar types, we need to focus only on
the leafs of the tree structure. To mark a domain we assign the type of the
abstraction to the leaf. Beacause we just need an information about abstract
elements, we do not store the rest of the leafs. Example abstract structure
with multiple domains is on the right picture.

\begin{center}
\begin{minipage}[t]{.47\textwidth}
\begin{tikzpicture}[->,>=stealth',level/.style={sibling distance = 2cm, level
    distance = 1cm}]
    \tikzstyle{fieldnode} = [
        ftreenode,
        minimum width = 0.2cm,
        minimum height = 0.2cm
    ];
    \node [fieldnode, label=180:{\small\code{Widget*}}] {}
        child { node [fieldnode, label=180:{\small\code{Widget}}] {}
            child{ node [fieldnode] {}
                child{ node [fieldnode, label=180:{\small\code{i64}}] {}
                edge from parent node [left, above=1pt] {\small\code{0}} }
                child{ node [fieldnode, label=0:{\small\code{Store*}}] {}
                    child{ node [fieldnode, label=0:{\small\code{Store}}] {}
                        child{ node [fieldnode] {}
                            child{ node [fieldnode, label=180:{\small\code{i64}}] {}
                            edge from parent node [left, above=1pt] {\small\code{0}} }
                            child{ node [fieldnode, label=180:{\small\code{i64}}] {}
                            edge from parent node [left] {\small\code{1}} }
                            child{ node [fieldnode, label=180:{\small\code{i64}}] {}
                            edge from parent node [right, above=1pt] {\small\code{2}} }
                        edge from parent node [left] {\small\code{0}}
                        }
                    edge from parent node [left] {\small\code{load}}
                    }
                edge from parent node [right, above = 1pt] {\small\code{1}}
                }
            edge from parent node [left] {\small\code{0}}
            }
        edge from parent node [left] {\small\code{load}}
        }
    ;
\end{tikzpicture}
\end{minipage}\hfill
\begin{minipage}[t]{.47\textwidth}
\begin{tikzpicture}[->,>=stealth',level/.style={sibling distance = 2cm, level
    distance = 1cm}]
    \tikzstyle{fieldnode} = [
        ftreenode,
        minimum width = 0.2cm,
        minimum height = 0.2cm
    ];
    \node [fieldnode, label=180:{\small\code{Widget*}}] {}
        child { node [fieldnode, label=180:{\small\code{Widget}}] {}
            child{ node [fieldnode] {}
                child{ node [fieldnode, label=180:{\color{orioles}zero}] {}
                edge from parent node [left, above=1pt] {\small\code{0}} }
                child{ node [fieldnode, label=0:{\small\code{Store*}}] {}
                    child{ node [fieldnode, label=0:{\small\code{Store}}] {}
                        child{ node [fieldnode] {}
                            child{ node [fieldnode, label=0:{\color{orioles}symbolic}] {}
                            edge from parent node [left] {\small\code{1}} }
                        edge from parent node [left] {\small\code{0}}
                        }
                    edge from parent node [left] {\small\code{load}}
                    }
                edge from parent node [right, above = 1pt] {\small\code{1}}
                }
            edge from parent node [left] {\small\code{0}}
            }
        edge from parent node [left] {\small\code{load}}
        }
    ;
\end{tikzpicture}
\end{minipage}
\end{center}

%\begin{center}
%\begin{minipage}[t]{.47\textwidth}
%\begin{minted}{cpp}
%struct Store {
%    int a, b, c;
%};
%
%struct Widget {
%    int data;
%    Store *store;
%};
%
%void foo() {
%    Widget w;
%    Store s;
%
%    w.store = &s;
%
%    _SYM int sym;
%    _ZERO int zero;
%
%    w.data = sym;
%    w.store->b = zero;
%    w.store->c = sym;
%}
%\end{minted}
%\end{minipage}\hfill
%\begin{minipage}[t]{.47\textwidth}
%\begin{minted}{llvm}
%%Widget = type { i32, %Store* }
%%Store = type { i32, i32, i32 }
%
%define void @foo() {
%  %w = alloca %Widget
%  %s = alloca %Store
%  %sym = alloca i32
%  %zero = alloca i32
%  ... ;incialization and annotations
%  %1 = load i32, i32* %sym
%  %2 = getelementptr %Widget, %Widget* %w, i32 0, i32 0
%  store i32 %1, i32* %2
%  %3 = load i32, i32* %zero
%  %4 = getelementptr %Widget, %Widget* %w, i32 0, i32 1
%  %5 = load %Store*, %Store** %4
%  %6 = getelementptr %Store, %Store* %5, i32 0, i32 1
%  store i32 %3, i32* %6
%  %7 = getelementptr %Widget, %Widget* %w, i32 0, i32 1
%  %8 = load %Store*, %Store** %7
%  %9 = getelementptr %Store, %Store* %8, i32 0, i32 2
%  store i32 %1, i32* %9
%}
%\end{minted}
%\end{minipage}
%\end{center}
%\add{ extend code border to the marginpar }
\end{example}

During the value propagation analysis, we store these trees for each variable.
To be consistent we represent the scalar types the same way (i.e. they are
represented as single node). For pointer types the trees look like a chain of
\code{load} edges.

In the future the tree-like representation may be easily used for alias analysis as proposed
in \cite{Rockai15}.

\add{ add wrap up of the section }

\subsection{Lifting of instructions}
Having the abstraction roots computed from value propagation analysis, we can
start program transformation. In this \LLVM pass we want all the instructions
working with the abstract values to be replaced by abstract intrinsics.
Additionally all the needed function prototypes have to be created, and
appropriate abstract implementation inserted into them.

From the abstraction roots computed in VPA we can do the transformation per
function. But first of all, the function prototypes are created. This is done
because during the transformation we may need to call some abstract function,
that has not been abstracted yet, but to call the function the prototype is
sufficient. This approach also elegantly deals with recursive calls.

Function prototypes are created according to the abstraction roots. Basically
the transformation iterates over all the sets of argument dependent roots and
the prototype is created based on the arguments included in the given set.
Moreover the return type is deduced from the reachability of the \code{ret} instruction
from the roots.

During this phase and further the transformation pass builds a mapping between
concrete types and abstract types. The abstract types is a \LLVM structure
expressing the abstracted concrete type and the domain. For example an integer
in symbolic domain is represented as follows.

\begin{minted}{llvm}
%lart.sym.i32 = type { i32 }
\end{minted}

Abstract types are always prefixed by \code{lart} abbreviation, followed by
domain name and finished by concrete \LLVM type name.

By this type system we can create desired prototypes for the functions from our
example program (\autoref{fig:exampleprogram}). Expecting that the \code{foo}
does not take any abstract value we need only to change to return type of the
signature. For the \code{bar} function there are two cases one does not take any
abstract value and second one have one abstract argument. After prototype
creation the resulting \LLVM declarations looks like this.
\begin{minted}{llvm}
declare %lart.sym.i32 @foo.2(i32, i32)
declare %lart.sym.i32 @bar.2(i32, %lart.sym.i32)
\end{minted}
Since \LLVM requires distinct names of the functions, the indexes are added in
the order of prototypes creation. The transformation is done per function,
concretely for each set of roots separately.

From the given roots we compute the reached set of instructions and sequentially
transform them. In order to have all the instructions transformed before their
result is used further in the function, we transform them in reverse postorder manner.

We will represent abstract operations as function calls, with similar naming
conventions as \LLVM-native intrinsic operations \cite{LLVM:langref}. The
operations still needs to track the domain type and the base \LLVM instruction.
For example for \code{add} operation we creates an abstract equivalent:
\begin{minted}{llvm}
%res = call %lart.sym.i32
            @lart.sym.add.i32(%lart.sym.i32 %a,
                              %lart.sym.i32 %b)
\end{minted}

When an instructions takes both concrete and abstract arguments we need to
transform the concrete value into correct domain. This is done by \code{lift}
operation which creates an abstract constant (described in
\autoref{sec:absdom}).

\marginpar{More unpleasant way exists, where combinatorial amount of abstract
operations is generated. For obvious reasons we did not take this path.}

Besides in domain transformations we need to solve inter-domain interactions. As
mentioned in \autoref{sec:interactions}, we need to transform in-domain booleans
into tristate manipulations. This has to be done to unify the control flow
handling, when dealing with multiple domains. Let's consider the following condition
to be abstracted.
\begin{minted}{llvm}
%cond = icmp sgt i32 %a, %b
br i1 %cond, label %then, label %else
\end{minted}
Since we do not know without further analysis whether the resulting value of
\code{icmp} is used in control flow determination, we prefer the result to
remain in the abstract domain. Only when we found a use of abstract boolean in
some control flow instruction (i.e. branching), we inserts conversion operation
from abstract boolean to tristate (described in \autoref{sec:interactions}).
Subsequently we lower the tristate into concrete boolean in order to preserve
explicit control flow:
\marginpar{During the execution, the lowering of tristate may call a non
deterministic choice if its value is uncertain i.e.~\emph{maybe}.}
\begin{minted}{llvm}
%cond = call %lart.sym.i1
             @lart.sym.icmp_sgt.i32(%lart.sym.i32 %a,
                                    %lart.sym.i32 %b)
%tris = call %lart.tristate
             @lart.sym.bool_to_tristate(%lart.sym.i1 %cond)

%bool = call i1 @lart.tristate.lower(%lart.tristate %tris)
br i1 %bool, label %then, label %else
\end{minted}

\begin{figure}
\begin{example}
With created prototypes the whole transformation of our example program from
\autoref{fig:exampleprogram}:

\bigskip
\resizebox{\textwidth}{!}{
\begin{tikzpicture}[>=stealth',shorten >=1pt,auto,node distance=4em, <->]
\tikzset{>=latex}
\tikzset{empty/.style = {minimum width=0cm,minimum height=1cm}}
    \node [llvm] (declx) {i32\textsubscript{sym} \textoverline{x}};
    \node [llvm, below = 0.2 cm of declx] (decly) {i32\textsubscript{sym}
    \textoverline{y}};
    \node [llvm, below = 0.2 cm of decly] (lifta) {\textoverline{a} = lift\textsubscript{sym}(a)};
    \node [llvm, below = 0.2 cm of lifta] (cond) { \textoverline{cond} = \textoverline{x} <\textsubscript{sym}
    \textoverline{a} };
    \node [br, below = 0.5 cm of cond] (br) { lower(\textoverline{cond}) };
    \node [llvm, below left = 0.5 cm of br] (lift4) { \textoverline{4} =
    lift\textsubscript{sym}(4) };
    \node [llvm, below = 0.2 cm of lift4] (brtrue) { \textoverline{y} =
    \textoverline{x} +\textsubscript{sym} \textoverline{4} };
    \node [llvm, below right = 0.5 cm of br] (liftbar) { y = bar(a, b) };
    \node [llvm, below = 0.2 cm of liftbar] (brfalse) { \textoverline{y} =
    lift\textsubscript{sym}(y) };
    \node [llvm, below = 2.5 cm of br] (retcall) { \textoverline{y} =
    \textoverline{bar}(a, \textoverline{y}) };
    \node [llvm, below = 0.2 cm of retcall] (ret) { ret \textoverline{y} };

    \begin{pgfonlayer}{background}
        \node [bb, fit = (declx) (decly) (lifta) (cond)] (bb1) {};
        \node [bb, fit = (brtrue) (lift4)] (bb2) {};
        \node [bb, fit = (brfalse) (liftbar)] (bb3) {};
        \node [bb, fit = (ret) (retcall)] (bb4) {};
        \node [fun, label={ i32\textsubscript{sym} foo(i32 a, i32 b)}, fit = (bb1) (bb2) (bb3) (bb4)] (fn) {};
    \end{pgfonlayer}

    \draw [flow, very thick] (cond) -- (br);
    \draw [flow, very thick] (br.west) -| (lift4.north) node [near start, above = 5pt] {true};
    \draw [flow, very thick] (br.east) -| (liftbar.north) node [near start, above = 5pt] {false};
    \draw [flow, very thick] (brtrue.south) |- (retcall.west);
    \draw [flow, very thick] (brfalse.south) |- (retcall.east);
\end{tikzpicture}
}
\bigskip

In the picture, symbolic values are represented by bar over them.
A notable change in the program are points where concrete values have to be
lifted to symbolic domain. Besides that a transformation of conditional
branching occur, with explicit branch over the lowered tristate. Lastly the
correct symbolic form of \code{bar} function has to be called. We can observe that call
with concrete arguments remains untouched.
\end{example}
\end{figure}

When dealing with aggregate or pointer types, we know that they are not
abstract, so we do not need to create abstract operations with them. Hence the
original instructions manipulating with aggregates remain untouched, except the
instructions that extract or inserts scalars to them.

In those situations, we take the use of results from the value propagation analysis.
Knowing which elements of aggregate are abstract, when the program tries to
reach to some of them (through some pointer) we just simply bitcast
the resulting concrete pointer to pointer to an abstract element. We can afford
to this because we know that in the particular memory location, an abstract value is
stored.

However this approach has its limitations, since the abstract representation has
to fit into the size of original type. As consequence we are not currently able
to store a pointer to representation of abstract type into smaller types then
pointer type (i.e. \code{i64} on 64-bit architecture)\marginpar{In \DIVINE is a
preparation in process, to be able to store an additional data into 'shadow'
memory. With it, \DIVINE will be able to maintain a pointer in the smaller type.}.

But as a big advantage of this approach is that we do not need to create an
abstract aggregate types and handcraft the whole structures. This way we also
can preserve the pointer arithmetic operations, since the offsets to the
aggregates do not change.

\add{ add wrap up of the section }

\subsection{Backwards constraint propagation and value restriction}
\label{sec:bcp}
When a branch is taken due to a lowered tristate, we need to restrict the values
engaged in the condition. Consider that branch is taken, based on comparison $a
> 10$, where $a$ is an abstract value. If we do not have any restriction on $a$
yet and branch was taken nondeterministically, as consequence the value of $a$
has to be greater than 10. However, when the branch was not taken,
the value has to be 10 or less. We will utilize this fact, to restrict the value
right after the branch was taken.

We can compute the value restrictions independently of the abstract domain, as
long as the abstract domain provides a right set of primitives. Concretely we
need to be able to infer a restricted value from a given value and some predicate
instruction on which we want to base the restriction, along with an actual
result of the predicate evaluation. For this purpose we have established
an \code{assume} operation:
\begin{minted}{llvm}
%x = call %lart.sym.i32
          @lart.sym.assume(%lart.sym.i32 %a,
                           %lart.tristate %cond,
                           i1 true)
\end{minted}
Meaning that value \code{\%x} carries the restricted value of \code{\%a}
according to the satisfied condition \code{\%cond}.

The insertion of assumes in two steps, first of all we restrict only the
tristates that are directly lowered and used in unconditional branching. By this
transformation we simply mark which path is taken based on the nondeterministic
choice.

Further we proceed in restriction of variables that are engaged in the condition
predicate. It may seem reasonable to carry on backwards in the program history to
provide more precise restriction, but backwards constraint propagation has some
limitations.

\add{ constrain propagation example }

Since the whole computation has to be done statically it is not clear how far the
backwards look up should proceed. The other limitation is that, the propagation
can only sufficiently analyze only the local static scope, and cannot infer
information from function arguments and global state.

Consequently we need to make trade-off between work given to domain
implementation and simple constrain analysis. Since the symbolic domain is
capable to acquire the implications resulting from simple restriction over the
branch condition we have decided to retain only simple assume insertion
algorithm for this thesis.

The symbolic domain can simply collect the restrictions as path condition and
restricts the possible variable evaluations based on the whole path
(in detail description of utilization of assumes in symbolic domain will be
described in \autoref{sec:symbolic}).

In addition to backwards analysis, assumes insertion demands a few minor
modifications to bitcode. When control flow of restricted value merges with
control flow from another restriction or original definition, we need to insert
appropriate $\varphi$ nodes.

For simpler dependency analysis and $\varphi$ nodes insertion, we have decided
to insert the restrictions on the edge between restricting branch and consequent
basic block. The insertion is implemented as creation of a new basic block
between two blocks redirection of the branch (illustrated in the
\autoref{ex:assumes}).

\begin{figure}
\begin{example} \label{ex:assumes}
Analyzing the example program, we need to restrict values depending on one
conditional branching.
\autoref{fig:exampleprogram}:

\bigskip
\resizebox{\textwidth}{!}{
\begin{tikzpicture}[>=stealth',shorten >=1pt,auto,node distance=4em, <->]
\tikzset{>=latex}
\tikzset{empty/.style = {minimum width=0cm,minimum height=1cm}}
    \node [llvm] (declx) {i32\textsubscript{sym} \textoverline{x}};
    \node [llvm, below = 0.2 cm of declx] (decly) {i32\textsubscript{sym}
    \textoverline{y}};
    \node [llvm, below = 0.2 cm of decly] (lifta) {\textoverline{a} = lift\textsubscript{sym}(a)};
    \node [llvm, below = 0.2 cm of lifta] (cond) { \textoverline{cond} =
    \textoverline{x} <\textsubscript{sym} \textoverline{a} };
    \node [br, below = 0.5 cm of cond] (br) { lower(\textoverline{cond}) };
    \node [llvm, fill=apple!40, below left = 0.5 cm of br] (asstrue)
    {\textoverline{x} = assume(\textoverline{x} <\textsubscript{sym} \textoverline{a})};
    \node [llvm, below = 0.5 cm of asstrue] (lift4) { \textoverline{4} =
    lift\textsubscript{sym}(4) };
    \node [llvm, below = 0.2 cm of lift4] (brtrue) { \textoverline{y} =
    \textoverline{x} +\textsubscript{sym} \textoverline{4} };
    \node [llvm, fill=apple!40, below right = 0.5 cm of br] (assfalse)
    {\textoverline{x} = assume(\textoverline{x} >=\textsubscript{sym}
    \textoverline{a})};
    \node [llvm, below = 0.5 cm of assfalse] (liftbar) { y = bar(a, b) };
    \node [llvm, below = 0.2 cm of liftbar] (brfalse) { \textoverline{y} =
    lift\textsubscript{sym}(y) };
    \node [llvm, below = 4 cm of br] (retcall) { \textoverline{y} =
    \textoverline{bar}(a, \textoverline{y}) };
    \node [llvm, below = 0.2 cm of retcall] (ret) { ret \textoverline{y} };

    \begin{pgfonlayer}{background}
        \node [bb, fit = (declx) (decly) (lifta) (cond)] (bb1) {};
        \node [bb, fit = (brtrue) (lift4)] (bb2) {};
        \node [bb, fit = (brfalse) (liftbar)] (bb3) {};
        \node [bb, fit = (asstrue)] (bbt) {};
        \node [bb, fit = (assfalse)] (bbf) {};
        \node [bb, fit = (ret) (retcall)] (bb4) {};
        \node [fun, label={ i32\textsubscript{sym} foo(i32 a, i32 b)}, fit = (bb1) (bb2) (bb3) (bb4)] (fn) {};
    \end{pgfonlayer}

    \draw [flow, very thick] (cond) -- (br);
    \draw [flow, very thick] (asstrue) -- (lift4);
    \draw [flow, very thick] (assfalse) -- (liftbar);
    \draw [flow, very thick] (br.west) -| (asstrue.north) node [near start, above = 5pt] {true};
    \draw [flow, very thick] (br.east) -| (assfalse.north) node [near start, above = 5pt] {false};
    \draw [flow, very thick] (brtrue.south) |- (retcall.west);
    \draw [flow, very thick] (brfalse.south) |- (retcall.east);
\end{tikzpicture}
}
\bigskip

In this simple case, proper assumes are inserted on the edges between branching
and consequent basic blocks. Remark that separate basic blocks were created for
the assumes. Creation of new basic blocks does not make any diffence in this
program, but for more complicated functions with loops edge splitting needs
to be done in order to enable $\varphi$ nodes creation.

We may see, that assumes return a new value for restricted variable, that is
used in following code. After an inspection we may notice that assume in
\emph{false} branch is unnecessery since the value of restricted
\code{\textoverline{x}} never used afterwards, but we don't do any optimizations
in this case.
\end{example}
\end{figure}

\add{ add wrap up of the section }

\subsection{Domain manipulations insertion}

After all the analysis and transformations are finished, we can proceed to actual
domain insertion. Since we now know all the instructions that have to be
transformed with relevant information about the domain, we do the substitution
per instruction without any further analysis.

Similarly as in the abstraction of instructions from the second phase we proceed in
the order, to have all the operations substituted before they are
used further in the bitcode (reverse postorder). When processing an
instruction, we look on the domain tag (in the name of the abstract intrinsic)
and accordingly choose the appropriate implementation. In current setting the
abstract intrinsics are replaced by calls to functions implementing the domain
transformers. This may optimized in the future as inlining of block of code or
direct replacing by single instruction (possible for simpler domain operations),
to minimize the number of call the bitcode.

From the domain implementation point of view, there are two things to be
provided. First, the implementation of transformers over the domain, including
the special ones --- \code{assume}, \code{lift} and \code{bool\_to\_tristate}.
And secondly, the implementation of lowering function, that given an abstract
\LART intrinsic and its arguments, produces a corresponding call to domain
transformer.

The limitation of whole transformation is that the abstract values must have
run-time representation using scalar values from concrete domain. This is not a
problem for simpler domains like $Z$, but for symbolic abstraction the value
representation has to be tweaked. As proposed in \autoref{sec:sym}, the data
have to be represented by tree-like structure. This representation can be then
stored behind a pointer, and stored on program heap. But by this approach we
introduce memory leaks to transformed bitcode. The other way is to copy symbolic
representation on every taken operation, what is quite inefficient in actual
bitcode runtime. Due to this we have used a special pointers provided by
\DIVINE (weak pointer), which \DIVINE recognize as non leaking. These pointers
are then maintained by \DIVM in simillar fashion as garbage collectors do.
Hence during the verification \DIVINE will distinguish the pointers to symbolic
values and conceal the memory leak.

\add{ add wrap up of the section }

\section{Symbolic state space exploration} \label{sec:symbolic}
When the program transformation is finished, abstracted program is given to
\DIVINE for model checking phase. Since the equality and reachability check
of the symbolic states cannot be done in only explicit manner, we need to employ
a similar exploration algorithm as \SymDIVINE (see \autoref{sub:symdivine}).

As described in previous sections, symbolic data in abstracted program are
described by data definitions (that are created by symbolic operations) and
path condition (that is defined by assumes taken during the program execution).
Then during the interpretation we need to check whether the state is
reachable, this is done by query on \SMT solver, whether the already taken
path condition is satisfiable. And secondly we need to distinguish whether we
have already seen the reachable state, that is done via equality check. Now we will
describe both queries in higher detail.

\subsection{Exploration algorithm}
The symbolic exploration algorithm, similarly as explicit algorithm, uses \DIVM
to interpret the \LLVM instructions. During the interpretation, abstracted
operations manipulates with symbolic values, according to the inserted implementation of
symbolic domain (i.e.~they operates on formula trees, see \autoref{sec:sym}).
Note that \DIVM has no clue about the symbolic representation and interprets
program as ordinary \LLVM bitcode.

The data definitions are currently represented as global data structure, that
keeps all the references to the formula trees. This structure has to be global
for abstract operations to be able to access it.

In order to create path condition, program needs to track constraints over the
symbolic values. This is achieved with \code{assume} calls inserted after the
control flow branching. Let's have a program with some branches:
\begin{minted}[linenos]{cpp}
__SYM int x;
if (x < 10) {
    if (x > 5) {
        ...
    }
}
\end{minted}
In the program above on the 4.~line we need to ensure that value of \code{x} is
between 5 and 10. This has to be done because the comparison on symbolic values
does not restrict the original value, only creates a nondeterministic choice
whether the branch is taken or not, hence value of \code{x} is arbitrary even if
we have decided that it is smaller then 10 in the first condition.

The restriction is achieved, right away after the branch is taken, by
\code{assume} call. To maintain the restrictions a path condition is stored in
some global data structure. Then, when the \code{assume} call is executed a
restriction is appended to the path condition. This is done in the
implementation of the \code{assume} call in the symbolic domain
(i.e.~manipulation with path codition is interpreted by \DIVM). Looking back to
the example, the path condition on the 4.~line is represented by two
restrictions: $(x < 10)$ and $(x > 5)$.

When a new state is generated we need to check whether the state is reachable,
this is done similarly as in \SymDIVINE, by checking the satisfiability of the
path condition. The path condition for \SMT query is created as conjunction of
all restrictions taken so far, i.e.~for line 4, the path condition looks like:
$(x < 10) \wedge (x > 5)$.

Since the path condition is changed only when the \code{assume} call is
interpreted, it is not necessary to check reachability for every state, but only
for those when the path condition changes, i.e.~after the \code{assume} call.

\subsection{Equality check}

Similarly as in \SymDIVINE (see \autoref{subsec:equality}), we will firstly
compare the explicit part of the states and then the symbolic part using the
\SMT solver.

Let's have two states $s_1$ and $s_2$, in the same control flow location, with
the same memory shape. This is when $s_1 = (c, m, d_1, \varphi)$ and $s_2 = (c,
m, d_2, \psi)$ for control location $c$, memory shape $m$, symbolic
definitions $d_1$, $d_2$ and path conditions $\varphi$ and $\psi$.

For example representing the memory configuration of two states it may look like
this:

\add{ pridat obrazok }



\bigskip

We will denote symbolic variables of state $s_1$ by the $v_1, \dots, v_2$
and $\widehat{v}_1, \dots, \widehat{p}_2$ for the state $s_2$. These variables
may have arbitrary value, hence they will be represented in the formula as
\emph{free} variables.

The values $p_1, \dots, p_k$ in the state $s_1$ and $\widehat{p}_1, \dots,
\widehat{p}_k$ in state $s_2$ are results of transformers over the symbolic
values (described by formula trees). From the logic point we consider them as
nullary functions, these functions return predicates over the symbolic values.

For example TODO
\add{ nularne fukcie na hodnoty }

Comparing these states in same control flow location, we compare the explicit
values (i.e.~$c_i = \widehat{c}_i$ for all the explicit values of the states).
When explicit part is the same we need to compare the symbolic part. To decide
whether the sets of concrete states represented by $s_1$ and $s_2$ are equal, we
define a formula $equals(s_1, s_2)$. The comparison is done via \SMT query:
\begin{equation*}
\begin{aligned}
equals(s_1, s_2) \defeq & ~(\varphi \iff \psi) \wedge \varphi \implies \\
                        & \forall~v_1,\dots, v_n (p_1 = \widehat{p}_1 \wedge \dots \wedge p_k = \widehat{p}_k )
\end{aligned}
\end{equation*}
The query says, that the states are equal when the path conditions are
equivalent and also satisfiable then the data definitions have to describe same
set. When the path conditions are equivalent, it does not matter which one we
check for satisfiability.

When path conditions are satisfied, we check for equality of data
definitions. We couple the corresponding symbolic values defined by nullary
functions and compare them for equality. These equalities have to be satisfied
for all free variables (symbolic inputs).

\add{ add chapter summary }
