\chapter{Preliminaries}\label{ch:preliminaries}

In this chapter, we will introduce foundations for proposed the
abstraction-based model checking approach. We will cover \LLVM bitcode as the
input language for verification. Focus will be on the instruction set for
value manipulation and its semantics, the \LLVM type system and the control flow
of programs. Subsequently the notion of model checking will be explored in
the context of \LLVM programs. The exposition of model checking is mainly based on
Petr Ročkai's dissertation \cite{Rockai15} further covered in \autoref{ch:divine}.

\section{Intro to \LLVMIR} \label{sec:introtollvm}
\marginpar{The \LLVM project is a collection of modular and
reusable compiler and toolchain technologies. Despite its name -- low level
virtual machine, \LLVM has little to do with traditional virtual
machines. The name \LLVM itself is not an acronym; it is the full name
of the project. \href{http://llvm.org/}{[llvm.org]}}

Besides defining the semantics of bitcode in the form of \LLVMIR (intermediate
representation), \LLVM{}~\cite{LLVM:web} is primarily a toolkit for building
compilers. It was first introduced by Ch. Lattner and V. Adve \cite{Lattner04}
as an optimization framework. Nowadays \LLVM represents a set of libraries for
building and manipulation of an intermediate representation of a program and
platform-specific code generators.

According to \LLVM language reference manual~\cite{LLVM:langref},
\LLVM is a static single assignment (\SSA) based representation that
provides type safety, low-level operations, flexibility, and
the capability of representing ‘all’ high-level languages cleanly. It is the
common code representation used throughout almost all phases of the \LLVM compilation
strategy.

Since the \LLVMIR simplifies semantics of the verified language, it is much more
suitable for model checking. Model checking can abstract from complicated
grammar of the verified language and process \LLVM bicode produced by the compiler
front-end.

\subsection{Instructions, Functions and other LLVM building blocks} \label{subsec:infnllvm}
An \LLVMIR aims to be slightly richer assembly like language. When we look
\marginpar{\LLVM programs are composed of modules, each of which is a
translation unit of the input programs.}
into \LLVMIR file (or \emph{module}), we can find there a description of global
data and function definitions. Each function definition consists of control flow graph
formed by basic blocks.

\begin{definition}\label{def:basicblock}
A basic block is a sequence of instructions that ends with a terminator
instruction (such as a branch or function return). The computation of a program
within a basic block is entirely sequential, performing instructions in their
order of appearance in the block~\cite{Rockai15}.
\end{definition}

Instructions from basic blocks manipulate values in virtual registers or
they move values between registers and memory. Each instruction has at most one
return value, which is assigned to register for the time of execution of a
function. Considering values in registers, a whole \LLVM bitcode is always in the
partial \SSA form.

\begin{definition}\label{def:ssa}
A static single assignment (\SSA{}) is a property of an intermediate representation
(\textsc{ir}), which requires that each variable is assigned exactly once, and every
variable is defined before it is used~\cite{Cytron91}.
\end{definition}

In addition to register values \LLVMIR allows address-taken variables (i.e. it's
allowed to take address of the variable and mutate it as normal variable).
Hence the address-taken variables are not part of the \SSA form. In unoptimized
\LLVM bitcode, each higher-level (C/\Cpp{}) variable becomes an address-taken
variable, created by an \code{alloca} instruction. Consequently, register values are only
created by the compiler as intermediate results of instructions. However, the \LLVM
optimiser can lift many address-taken variables into registers if the addresses
of those variables are not actually taken.

For our purpose we will meet only a subset of \LLVM instructions. Concretely
instructions for manipulation of transformed values, memory operations and
control flow (terminator) instructions. Precise documentation of all can be
found at \LLVM language reference manual \cite{LLVM:langref}. For better
undertanding of \LLVM bitcode, look at the following example.

\improvement{ Add hrule to examples }

\begin{example}
Let's have a look at piece of \Cpp{} code and corresponding \LLVM bitcode.

\begin{minted}{cpp}
int factorial(int n) {
    if (n == 0 )
        return 1;
    return n * factorial(n - 1);
}
\end{minted}

\begin{minted}{llvm}
define i32 @factorial(i32 %n) {
  %1 = icmp eq i32 %n, 0
  br i1 %1, label %then, label %else
; <label>:then:
  ret i32 1
; <label>:else:
  %2 = sub nsw i32 %n, 1
  %3 = call i32 @factorial(i32 %2)
  %4 = mul nsw i32 %n, %3
  ret i32 %4
}
\end{minted}

\noindent
In \LLVM bitcode, the \mintinline{cpp}{factorial} function is defined as a global object (prefixed by \texttt{@})
\marginpar{To distinguish between register values and mutable
address-taken variables \LLVM use two types of prefixes, \texttt{\%} for register values
and \texttt{@} for mutable and global values.}
\code{@factorial} with return type \code{i32} (a 32 bit integer) and an argument called \code{\%n},
also of type \code{i32}. In the function implementation we can see almost direct
correspondence to \Cpp{} code. The function is logically divided into a few basic
blocks annotated with \code{<label>} and the name of the basic block (see~\autoref{def:basicblock}). First basic block, also called \emph{entry} block,
contains equality comparison (\code{icmp} instruction) of the input argument \code{\%n}
and \code{0}, followed by conditional branching instruction \code{br}.

The branching instruction connects basic blocks and determines control flow depending
on the value of register \code{\%1}. Branching jumps to basic block labeled with \code{\%then} if
the condition \code{\%1} was true, otherwise it jumps to \code{\%else} basic block. Besides
branching, \LLVM uses \code{br} instructions to represent loops, by jumping
backwards in the control flow graph.

\noindent
In basic block \code{\%then} one is returned with \code{ret} instruction. In the \code{\%else} branch,
besides arithmetic operations (\code{sub} representing subtraction and \code{mul} multiplication),
a recursive call is made using a \code{call} instruction.
\end{example}

\subsection{Type system} \label{subsec:typesystem}

An observant reader may have noticed that \LLVMIR is a strongly typed language.
Besides the \code{void} type, \LLVM distinguishes basic single value types, such as
\code{iN} where \code{N} determines bit width of integral type. We have already met \code{i32}
as 32-bit integer and \code{i1} usually used for booleans.
\LLVM additionally supports floating point types as \code{float} and \code{double} and also C-like pointer types written as \code{<type>*}.

In order to support aggregate types, \LLVM recognizes array types, struct types
and vector types \cite{LLVM:langref}. Similarly as in C, array type represents a consecutive block
of memory of same types, e.g.~to denote array of 10 integers \LLVM writes
\code{[10 × i32]}. To represent a collection of data members together in memory \LLVM
defines a structure type. \LLVM identifies struct types with given
identifiers in the global namespace as follows for a structure called \code{aggregate}
consisting of \code{i32} and a pointer to pair of \code{i32} values:

\begin{minted}{llvm}
%aggregate = type { i32, { i32, i32 }* }
\end{minted}

Structures in memory are accessed using \code{load} and \code{store} instructions.
A \code{getelementptr} instruction is used to compute an offset into an aggregate type.
Structures in registers are accessed using the \code{extractvalue} and \code{insertvalue}
instructions \cite{LLVM:langref}.

In order to cast between types \LLVM provides a variety of casting instructions.
To be specific, a \code{bitcast} instruction for explicit type casting, \code{ptrtoint} and
\code{inttoptr} for conversions between pointers and integers and \code{trunc}, \code{zext},
\code{sext} for transformation to larger or smaller integer types.

\subsection{Control flow} \label{subsec:controlflow}

As mentioned earlier, \LLVM supports control flow instructions like \code{br} for
conditional and unconditional jumps between basic blocks. To direct control flow
between functions, \LLVM provides \code{ret} for returning from functions, \code{call} and
\code{invoke} for calling a function. Except \code{call}, all control flow instructions
are always at the end of a basic block, hence they are called \emph{terminator}
instructions.

As consequence of branching control flow graph and \SSA form, \LLVM needs to
support acquisition of value from different basic blocks. For this
purpose a special instruction is designed -- a \code{phi} node \cite{Cytron91}.

\begin{minted}{llvm}
%v = phi i32 [ %a, %then ], [ %b, %else ]
\end{minted}

As a result of \code{phi} node instruction, \code{\%v} contains either value \code{\%a},
if the execution came from basic block \code{\%then}, or value \code{\%b} for execution
coming from \code{\%else} basic block.

\subsection{Semantics} \label{subsec:semantics}

In order to build abstraction over \LLVM bitcode, we need to setup a good
semantics model of the instruction set. For this purpose, small-step operational
semantics \cite{Plotkin04} is a good candidate.

\begin{definition}
Operational semantics expresses the evaluation of commands as a relation between
a command, initial state and a final state.
\end{definition}

\noindent
Since \LLVMIR semantics prevents unnamed values creation, we don't need to
bother by expression semantics. The only state mutators are instructions, which are
semantically atomic and operate only on existing named values. Moreover
instructions may mutate at most single value from program state. Hence, the
semantic function with derivation rules can be derived from \LLVM
language reference \cite{LLVM:langref}. Program state $\sigma$ will basically
consists of two parts -- stack frames and memory. Stack frames represent
register values, similarly in memory part are stored allocated variables.
Further description of used \LLVM semantics in \DIVINE can be found in Ročkai's
disertation thesis \cite{Rockai15}.

\section{\LLVM compilation process} \label{sec:compilation}

Compilation process of \LLVM tightly depends on a given \emph{frontend} compiler that produces
\LLVMIR, e.g.~\clang for C/\Cpp{}, see \autoref{fig:compilation}.
\LLVM bitcode generated by the frontend is then processed and optimized by \LLVM.
Optimization is divided into \LLVM \emph{passes}, where each of the passes
performs some optimization task or code analysis. Passes can be run directly
in the compiler or they can be run by an external tool, e.g.~the \LLVM
optimization tool \emph{opt} \cite{LLVM:opt}. After optimization, \LLVMIR is passed to
an appropriate code generator, which generates assembly bitcode for a specific
platform, e.g.~\texttt{x86} or ARM.

\begin{figure}[!ht]
\centering
\resizebox{\textwidth}{!}{
\begin{tikzpicture}[>=stealth,shorten >=1pt,auto,node distance=4em, <->]

    \tikzstyle{component}=[draw, text centered, rounded corners=1pt, minimum height=2.8 em, minimum width=2 cm, text width=1.8 cm]
    \tikzset{>=latex}
    \tikzset{font={\fontsize{9pt}{12}\selectfont}}

    \node [component] (clang) { Clang C++ frontend};
    \node [component, below = 0.5 cm of clang] (gcc) { llvm-gcc frontend};
    \node [component, below = 0.5 cm of gcc] (ghc) { GHC frontend};
    \node [left = 0.4 cm of clang](c++) {C++};
    \node [left = 0.4 cm of gcc](fortran) {Fortran};
    \node [left = 0.4 cm of ghc](haskell) {Haskell};

    \node [component, right = of gcc] (opt) {\LLVM optimizer};

    \node [component, right = of opt] (power) {\LLVM PowerPC backend};
    \node [component, above = 0.5 cm of power] (x86) {\LLVM \texttt{x86} backend};
    \node [component, below = 0.5 cm of power] (arm) {\LLVM ARM backend};

    \node [right = 0.4 cm of x86] (rx86) {\texttt{x86}};
    \node [right = 0.4 cm of power] (rpower) {PowerPC};
    \node [right = 0.4 cm of arm] (rarm) {ARM};

    \draw [->] (c++) -- (clang);
    \draw [->] (fortran) -- (gcc);
    \draw [->] (haskell) -- (ghc);

    \draw [->] (clang.east) -| ([xshift= 0.3 cm]gcc.east) -- (opt.west);
    \draw [->] (gcc) -- (opt) node [pos=0.6,above] {\tiny \LLVMIR};
    \draw [->] (ghc) -| ([xshift= 0.3 cm]gcc.east) -- (opt);

    \draw [->] (opt) -- ([xshift= -0.3 cm]power.west) |- (x86);
    \draw [->] (opt) -- (power) node [pos=0.4,above] {\tiny \LLVMIR};
    \draw [->] (opt) -- ([xshift= -0.3 cm]power.west) |- (arm);

    \draw [->] (x86) -- (rx86);
    \draw [->] (power) -- (rpower);
    \draw [->] (arm) -- (rarm);
\end{tikzpicture}
}
\caption{\LLVM compilation process consists of 3 steps -- parsing of the input
language and producing \LLVMIR in the frontend part, optimization of produced
\LLVMIR and generation of platform specific assembly by compiler backend.}
\label{fig:compilation}
\end{figure}

\section{Model checking} \label{sec:mc}
Model checking is an automated technique that, given a finite-state model of a
system and a formal property, systematically checks whether this property holds
for (a given state or edge in) that model \cite{Baier08}. To define a model checking problem,
we need set up a transition system that describes model of a system to be
verified.

In \DIVINE, we will talk concretely about transition based model checking
via automata-theoretic approach \cite{Vardi96}. Generally in automata based model checking,
a model of program is represented by transition system $T$ induced by program
transition relation, and property $\varphi$ is transformed to transition based
automaton $A_\varphi$. Model checking algorithm is then executed on product
of $T$ and $A_\varphi$, further described by Duret in \cite{Duret04}.

\begin{definition}\label{def:ts}
A transition system $TS$ is a tuple $(S, s_0, Act, \longrightarrow, AP, L)$, where
\begin{itemize}
    \item $S$ is a set of states,
    \item $s_0$ is a initial state,
    \item $Act$ is a set of actions,
    \item $\longrightarrow \subseteq S \times \mathrm{Act} \times S$ is a
    transition relation,
    \item $AP$ is a set of atomic propositions, and
    \item $L \colon S \rightarrow 2^{AP}$ is a labeling function.
\end{itemize}
\end{definition}
\noindent
For readability, we will write $s \xrightarrow[]{a} s'$ instead of $(s, a, s') \in
\longrightarrow$. To describe transition system in context of \LLVM program we
will consider a program state as a configuration of memory and stack variables.
Change of program state is achieved only by instructions that modify or
introduce new program variables. Hence transition label $l \in Act$ is sequence
of \LLVM instructions ending with a visible action.

To check whether state $s$ satisfies property $\varphi$, we want show that a set of
atomic propositions induced by $L(s)$ satisfies property $\varphi$, i.e.:
$$ s \models \varphi \iff L(s) \models \varphi.$$
Now we have everything to set up a model checking problem.

\begin{definition}\label{def:mc}
Let $M$ be a transition system and $\varphi$ a
formula of a temporal logic (i.e., the specification). Find all states $s \in M$
such that $s \models \varphi$.
\end{definition}

Since our proposed abstractions of the program should not interfere
with a proposition checking, we will consider model checking only as technique for
checking error state reachability (safety). From \LLVM point of view, error states are
those that do not satisfy some program assertion (i.e., the transition to
those states were made through unsatisfied assert call).

\subsection{Abstraction model checking} \label{subsec:amc}

When the number of states is large, it is difficult to determine whether a program
is correct. To reduce state space size, an abstract model checking technique was proposed
by E. Clarke et al.~in \cite{Clarke94}.

Abstract model checking abstracts a given model, taking some concrete state,
which is an $n$-tuple of variables, and producing a $n$-tuple of abstract
variables. Abstraction of state $s = (s_1,\dots,s_n)$ is then formed by surjections
$\alpha = (\alpha_1,\dots,\alpha_n)$ that map each $s_1$ to abstract variable $\widehat{s}_1$.
Applying this transformation to transitions system of a program
results into a \emph{minimal abstraction} $\widehat{M}$ of the original program.

Formally, given a concrete domain $C$, we define for a state $s \in C_1 \times \dots \times C_n$
surjections $\alpha_1,\dots,\alpha_n$, that map each variable of the state to the abstract domain
$A$, i.e. $\alpha_i : C_i \rightarrow A_i$. These mappings induce
a surjection $\alpha: C \rightarrow A$ defined as:
\[\alpha(s_1,\dots, s_n) = (\alpha_1(s_1), \dots , \alpha_n(s_n)).\]

\noindent
Alternatively we may define a relation between abstracted and concrete values by
a set of equivalence relations. Concretely each $\alpha_i$ corresponds to the
equivalence ${\sim}_i \subseteq C_i \times C_i$ defined by
$$s_i  \sim s'_i \iff \alpha_i(s_i) = \alpha_i(s'_i).$$

The mapping $\alpha$ induces an equivalence relation $\sim \in C \times C$ similarly.
We can now define what it means for transition system to be transformed to
abstract version. Lets $M$ be a transition system over some concrete domain $C$ and
$\alpha$ be a surjection from $C$ to abstract domain $A$. Intuitively we want an
abstract state $\widehat{s}$ to represent all states from $M$ for which $\alpha(s) =
\widehat{s}$.

\begin{definition}\label{def:am}
    Let $\widehat{M}$ be a transition system over $A$. We say that $\widehat{M}$
    over-approximates $M$ when
    \begin{enumerate}
        \item $\alpha(s_0) = \widehat{s}_0$ is initial state in $\widehat{M}$, and
        \item $\exists s \exists p \exists a . \alpha(s) = \widehat{s} \wedge \alpha(p) = \widehat{p} \wedge s
               \xrightarrow[]{a} p$ implies $\widehat{s}
               \xrightarrow[]{a} \widehat{p}$ in $\widehat{M}$.
    \end{enumerate}
\end{definition}

\noindent
The transition system that has only transitions required by \autoref{def:am} is called
\emph{minimal}.

Once we have an abstract transition system $\widehat{M}$, the model checking is
straightforward. Traversing the abstract state space is done in the same manner as
in original model checking approach. But satisfaction of property has to be checked
over abstract state. Since abstract states represent a set of concrete states,
it may happen that abstraction does not separate error transitions from the rest.
Hence it may announce an error even though the error transition is not reachable.
But since abstraction includes original model $M$ we know that if $\widehat{M} \models \varphi$
then $M \models \varphi$, for some property $\varphi$. If the abstraction satisfies
also this property $\widehat{M} \not\models \varphi \implies M \not\models \varphi$,
we are saying that abstraction is \emph{precise}.

\change{ Change description fo model checking from statet based to transition based }
\add{ Zmienit automata based approach }
\add{ Safety je pod LTL }
\change{ Domysliet abstrakciu stavov tak aby zodpovedala sym abstrakcii }
