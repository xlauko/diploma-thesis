\chapter{DIVINE and friends}\label{ch:divine}

In this chapter we will look under the hood of DIVINE~\cite{Divine17} the model checker.
And on top of that we will try to tackle problem of handling inputs by
symbolic model checking as introduced in~\cite{Barnat14}. We will try to emphasize the
differences between classical model checking algorithm and symbolic algorithm,
in order to decompose symbolic part in following chapter.

For purpose of this thesis only minor knowledge of \DIVINE architecture is
needed. Our interest is mostly in \LLVM interpreter, since its only part that
interacts with a transformed program. Deeper description of \DIVINE can be found
in \cite{Divine17} or on project website
\href{https://divine.fi.muni.cz/}{divine.fi.muni.cz}.

\section{Model checking with \DIVINE}

\marginpar{Described architecture is of latest release \DIVINE 4.0.}

\DIVINE is a modular platform for verification of real world programs.
Overall architecture can be divided into 2 parts: a verification environment
that provides tools for \LLVM interpretation and state space exploration, and a
runtime environment, whose purpose is to give support of language features like
memory allocation, threads and standard libraries.

The verification and runtime environment are split into several components with
precisely defined interfaces between them, see figure \autoref{fig:architecture}.

The runtime part consists of user's program accompanied with necessary
libraries. For user's program is provided a \Cpp{} standard library and threading
library suited for \DIVINE interpreter. As environment for program is provided
a \DIVINE operating system (\DIOS) that provides support for thread management
and scheduling. For communication between program and operating system a set
of `syscalls` is provided by \DIOS.
The verification part consists of 2 main components: a \DIVINE virtual machine
(\DIVM) that interprets \LLVM bitcode and provides a state space generator, and
verification core that takes care of verification procedure.

\begin{figure}[!ht]

\resizebox{\textwidth}{!}{

\begin{tikzpicture}[>=stealth,shorten >=1pt,auto,node distance=4em, <->]

\tikzstyle{bcomponent} = [
     color=pruss,
    fill=white,
    thick,
    draw,
    text centered,
    minimum height= 1cm,
    minimum width=2.2 cm,
    text width=6 cm,
];

\node [bcomponent] (input) {User's program $+$ libraries};
\node [clabel, above = 0cm of input] (renv) {Runtime environment};
\node [bcomponent, below = 0cm of input] (runtime) {\Cpp{} standard libraries, \texttt{pthreads}};
\node [fnlabel, right = 1.7cm of runtime] (syslabel) {syscalls};
\node [left = 2cm of runtime] (divine) {\large\DIVINE};
\node [bcomponent, below = 0cm of runtime] (dios) {\DIOS};

\node [clabel, below = 0.4cm of dios] (venv) {Verification environment};
\node [fnlabel, left = 2.7cm of venv] (hyplabel) {hypercalls};
\node [bcomponent, below = 0cm of venv] (divm) {\DIVM};
\node [bcomponent, below = 0cm of divm] (vc) {Verification core};


\begin{pgfonlayer}{background}
     \node[runtime, outer, fit = (renv) (input) (runtime) (dios)] (runtimebox) {};
\end{pgfonlayer}

\begin{pgfonlayer}{background}
  \node[verification, outer, fit = (venv) (divm) (vc)] (verificationbox) {};
\end{pgfonlayer}

\draw [-,dashed, very thick, color = pruss] ([xshift=4cm]input.south east) -- ([xshift=-4cm]input.south west);
\draw [flow,rectangle connector=1.5cm] (input.east) to (dios.east);
\draw [flow,rectangle connector=0.75cm] (runtime.east) to (dios.east);

\draw [flow,rectangle connector=-1.5cm] (runtime.west) to (divm.west);
\draw [flow,rectangle connector=-0.75cm] (dios.west) to (divm.west);
\end{tikzpicture}
}

\caption{\DIVINE architecture is divided into two parsts. A runtime environment
represents a \LLVM bitcode that is interpreted by \DIVM. And verification
component that provides model checking tooling. A communication between
layers is done by \texttt{syscalls} to \DIOS and by \texttt{hypercalls}
to \DIVM. We may notice that user's program can not call \texttt{hypercalls}
and communicates only with OS layer.}\label{fig:architecture}
\end{figure}

\subsection{\DIVINE virtual machine (\DIVM)}

A \DIVINE virtual machine aims to provide a bare minimum for \LLVM-based
model checking. This involves an execution of instructions, memory management,
nondeterministic choice and tracking of atomic sections. A deeper description of
\DIVM can be found in \cite{RockaiCB17}.
\marginpar{Nondeterministic choice serves for simulation of threads interleaving
and potential input generation.}

Side by side of \LLVM bitcode evaluation, \DIVM stores a representation of a
program state. A snapshot of a state can be passed to verification algorithm
for safety analysis. And vice versa a verification algorithm may ask \DIVM for
successors of a given state.

A program state is represented by memory configuration described by graph.
Nodes of graph represents objects (e.g.~results of allocation, global
variables) and edges represents pointers between this objects.
\DIVM stores these states compressed and hashed, and compares the graphs
directly for state equality.

\DIVM in \DIVINE is accompanied by runtime environment, which is executed on top
of \DIVM. This environment is expected to provide a \emph{scheduler}, that is invoked
by \DIVM during generation of state successors. An interface between \DIVM and
runtime environment is provided by \emph{hypercalls}, that enable runtime to modify
memory and create a state space branching by nondeterministic choice.

\begin{example}
A nondeterministic choice provided by \DIVM as \texttt{\_\_vm\_choose} function
gives an ability to branch a state space. For example following nondeterministic
choice branches state space for each \texttt{n} from a given \texttt{range}.

\begin{minted}{cpp}
int n  = __vm_choose( range );
\end{minted}

\end{example}
\subsection{State space reductions}

In order to enable verification of real world programs \DIVINE needs to employ
some state space reduction technique. For this purpose a so called
$\tau$-reduction is implemented, to eliminate superfluous intermediate states,
and heap symmetry-based reduction \cite{Rockai13} \cite{RockaiCB17}. It is demanded
that state space reductions preserve all safety and \LTL properties verified
by \DIVINE.

In \LLVM, a large subset of instructions have no observable effect for other
threads. This holds for instructions that manipulates with registers (registers
are private for the executed function) and those instructions, that manipulates
only with private memory of a thread.

Hence \DIVINE does not need to emit a new state on every imterpreted
instruction, but only when an observable action is reached. In resulting state
space, edges correspond to sequences of non observable instructions.
Alternatively \DIVINE supports atomic blocks and these are interpreted as sequence
of non observable instructions.

In \DIVM, a support for the $\tau$-reduction is implemented by \code{interrupt\_mem}
and \code{interrupt\_cfl} hypercalls. These calls enables a signalization of an
observable action to state space generation and forces an interrupt in edge
generation. The \code{interrupt\_mem} hypercall signals to \DIVM that a memory
operation is about to be executed. The \code{interrupt\_cfl} is used for signalization
of potential loop in the program state space.

\subsection{Verification workflow}

A verification in \DIVINE is split into two phases, see figure
\autoref{fig:verification}. A preprocessing phase, where an input program is
transformed into suitable input for \DIVM. In second phase a transformed program
is processed by \DIVM and some verification algorithm.

Since most of this thesis extends the preprocessing part, let's have a closer
look into program transformations. The transformations are similar to \LLVM
optimization passes (they work in \LLVM -to-\LLVM manner). They modify an input
program in order to extend a model checker capabilities. For example a
verification of programs with weak memory models is done via a transformation
\cite{Still16}, verification of programs with exceptions \cite{Still17} and minor
optimizations like interrupts insertion for faster scheduling.

Transformations are made by external tool named \LART introduced in~\cite{Rockai15}
as \LLVM Abstraction \& Refinement tool. The main motivation behind \LART is to
provide a preprocessing for \LLVM -based verification tools, simplifying their
job by reducing the problem size without compromising soundness of the
verification.

As abstraction tool \LART was never fully implemented and till this thesis it
was meant only as a proof-of-concept. The main aim of this thesis is to provide
a core analysis for \LART to be able to inject an arbitrary abstraction into a
program.

\begin{figure}[!ht]
\centering

\begin{tikzpicture}[>=stealth',shorten >=1pt,auto,node distance=4em, <->]

\tikzset{>=latex}

  \node [color=pruss] (input) {\Cpp{} code};
  \node [color=pruss, align = center, right = 2.5 cm of input.south, anchor = south, text width = 2.1cm] (prop) {property and options};

  \node [component, text width = 2.7 cm, below = 1 cm of input] (cc) {Compiler};
  \node [component, text width = 2.7 cm, below = 0.7 cm of cc] (instr) {LART};
  \node [emptycomponent, dashed, thick, left = 0.5 cm of cc] (runtime) {\DIOS and libraries};
  \node [clabel, above = 0.1 cm of runtime] (preproc) {Preprocessing};

  \node [component, below = 2.5 cm of instr, text width = 2.7cm] (interpreter) {Interpreter};

  \node [component, left = 0.5 cm of interpreter, text width = 2.7cm] (gen)
  {State space generator};

  \node [clabel, above = 0.9 cm of gen.west, anchor = west] (divml) {\DIVM};
  \node[emptycomponent, dashed, fit = (gen) (interpreter) (divml)] (divm) {};
  \node [component, text width = 3 cm, below = 0.5 cm of gen] (alg) {Verification algorithm};
  \node [clabel, above = 1.3 cm of divm.west, anchor = west] (verif) {Verification};

  \node [below = 1 cm of alg, text=apple, anchor = west] (valid) {Valid};
  \node [below = 1 cm of alg, text=orioles, anchor = east] (err) {Error};

  \begin{pgfonlayer}{background}
      \node[runtime, outer, fit = (cc) (runtime) (preproc) (instr)] (prepbox) {};
  \end{pgfonlayer}


  \begin{pgfonlayer}{background}
      \node[verification, outer, fit = (verif) (interpreter) (alg) (divm) (divml)] (prepbox) {};
  \end{pgfonlayer}

  \draw [flow, dashed] (input) -- (cc);
  \draw [flow] (cc) -> node [font=\small, left = 0cm, midway] {LLVM IR} (instr);
  \draw [flow] (runtime) -- (cc);
  \draw [flow, dashed] (prop) |- (instr);
  \draw [flow, dashed] (prop) |- (interpreter);
  \draw [flow] (instr) -> node [font=\small, left = 0cm, near start] {\DIVM IR} (interpreter);

  \draw[flow,<->] (interpreter) -- (gen);
  \draw[flow,<->] (alg) -- (gen);

  \draw [flow, dashed] (alg.south -| valid.north) -- (valid.north);
  \draw [flow, dashed] (alg.south -| err.north) -- (err.north);
\end{tikzpicture}

\caption{Verification process in \DIVINE consists of preprocessing and
the mere state space exploration. In preprocessing step an input code is
compiled and linked with \DIOS runtime and \DIVINE support
libraries. Then the bitcode is instrumented by \LART producing a suitable \LLVM bitcode with \DIVM hypercalls (called \DIVM IR). And finally the bitcode is interpreted in \DIVM and the verification result is produced.}\label{fig:verification}
\end{figure}

\subsection{Symbolic model checking with \SymDIVINE}

Since \DIVINE is an explicit state model checker, its big pitfall are inputs. The
only way to handle inputs in \DIVINE is to enumerate all possibilities, what
leads to enormous state space explosion. Hence current \DIVINE is usable only on
closed programs (programs without input). As an attempt to solve input problem a
\SymDIVINE was designed as an extension of \DIVINE. \SymDIVINE is
a proof-of-concept tool that is based on idea of \emph{control-explicit data-symbolic}
model checking \cite{Barnat14}.

\subsection{Control-explicit data-symbolic model checking}

In compare to exhaustive enumeration of states by purely explicit approach, a
control-explicit data-symbolic approach tries to group states into sets, when
they differ only in data values but not in control location. In this way \SymDIVINE is able to
simulate inputs as sets of possible values. These sets, also called \emph{multi states}, are described by explicit control location and symbolic representation of
data, in our case by quantified \emph{smt} bit-vector formula (for better description see section
\autoref{sec:multistates}). During model checking a state space exploration
algorithm works directly with multi states, hence the computation is more
time-consuming, since some \SMT solver has to be called in order to
distinguish reachable states. On the other hand symbolic approach may bring
exponential memory savings and avoid state space explosion caused by inputs.

\subsection{Representation of multi states} \label{sec:multistates}

A challenging part of symbolic model checking is how to define a suitable
representation of states. Representation of control location (explicit part of
multistate) is straightforward. A model checker just needs to store
a program location for each thread. On the other hand coming up with a good
representation of symbolic data may be quite challenging. Since \DIVINE aims
for bit-precise verification a suitable choice is representation of data
by \emph{smt} bit-vector formulae~\cite{Bauch14}.

\marginpar{ \color{red} TODO introduce \SMT }

\SymDIVINE does not support dynamic memory allocation, hence representation
of symbolic data becomes much more simpler. In current state representation of
data is done by quantified bit-vector \SMT formulae. An unambiguous identification
of symbolic variables is crucial. Therefore \SymDIVINE uses a names given by function
segment on stack and offset of given variable in the corresponding segment.

Besides remembering current values of variables, \SymDIVINE has to keep for
some variables previous evaluations (also called generations). This happens when
value of symbolic variable depends on another symbolic variable, see example
\ref{ex:gen}.

\begin{example}\label{ex:gen}
Having following C code \SymDIVINE has to store multiple generations of variable
\code{a}, because after evaluation of all statements \SymDIVINE needs to know
that, \code{b} is equal to \code{a} from first line (first generation of
\code{a}) and \code{a} (in second generation) is equal to $0$.

\begin{minted}{cpp}
int a = 10;
int b = a;
a = 0;
\end{minted}

\end{example}

A symbolic representation of data is further structured into two parts -- a
program \emph{path condition} and data \emph{definitions}. The path condition is a
conjunction of formulae that represents a restriction of the data that have been
collected during the branching along the path leading to the current location.
Definitions, on the other hand, are made of a set of formulae in the form
\emph{variable = expression} that describe internal relations among variables.
Definitions are produced as a result of an assignment or arithmetic
instructions. The structure of symbolic data representation allows for a
precise description of what is needed for the model checking, but it lacks the
canonical representation. As a matter of fact, the equality of multi states
cannot be performed as a syntax equality, instead, \SymDIVINE employs \SMT
solver and quantified formulae to check the satisfiability of a path condition
and to decide the equality of two multi states \cite{Mrazek16}.

\subsection{State space exploration}

When program reads some input, an explicit state model checker emits a
new successor for every possible input value. It is good to remark that all of
these states differ only in a single value, but they occupy same control flow
location. Hence a same sequence of instructions is applied to them.

\marginpar{Nondeterministic input values for \SymDIVINE are introduced by
\texttt{\_\_VERIFIER\_nondet\_\{type\}}, where \texttt{\{type\}} defines type of
input value.}

On the other hand a symbolic approach emits only one multi state. We may look at
multi state as it represents a set of purely explicit states (defined by
set-based reducion~\cite{Havel14}). See an example \ref{ex:reduction} for comparison of
explicit state space and state space reduced by set-based reduction.

The only event that splits a state space of multi states is a control flow branching,
that depends on a symbolic value. This happens when a symbolic value is used in
some comparison (in \LLVM an \code{icmp} instruction). Consequently a pair of
multi states has to be generated, one that corresponds to state when the
comparison result was true and second when the result was false. Additionall
according to comparison result corresponding values in multi states have
to be constrained according to comparison condition. For better explanation see
following example \ref{ex:reduction}.

\begin{example}\label{ex:reduction}
Let's have a look on impact of set-based reduction on program with input and
branching. A corresponding \LLVM bitcode is on right side.

\begin{center}
\begin{minipage}[t]{.47\textwidth}
\begin{minted}{cpp}
unsigned int a = input();
if (a >= 65535) {
    ...
} else {
    ...
\end{minted}
\end{minipage}\hfill
\begin{minipage}[t]{.47\textwidth}
\begin{minted}{llvm}
%a = call i32 @input()
%b = icmp uge i32 %a, 65535
%br i1 %b, label %t, label %e
\end{minted}
\end{minipage}
\end{center}

\bigskip
\noindent
In order to precisely verify a program with input \DIVINE, has to generate all
possible evaluations of \code{input} call. Hence the resulting state space suffers
from great state space explosion. We may notice that state space is branched when
the input call is executed, but following execution continues in sequential
manner.

\begin{centering}
\resizebox{\textwidth}{!}{
\begin{tikzpicture}[]
    \tikzstyle{every node}=[align=center, minimum width=1.25cm, minimum height=0.6cm]
    \tikzset{empty/.style = {minimum width=0cm,minimum height=1cm}}
    \tikzset{dots/.style = {draw=none}}
    \tikzset{>=latex}
    \node [pc, text width = 1 cm] (s) {$\bot$};
    \node [right = 3cm of s] (mid) {};

    \node [pc, above = -0.25 cm of mid, minimum width=2.7cm] (s65534){a = 65534};
    \node [pc, below = -0.25 cm of mid, minimum width=2.7cm] (s65535){a = 65535};

    \node [dots, above = 0 cm of s65534] (dots1){\LARGE$\vdots$};
    \node [dots, below = -0.2 cm of s65535] (dots2){\LARGE$\vdots$};

    \node [pc, above = -0.2 cm of dots1, minimum width=2.7cm] (s0) {a = 0};
    \node [pc, below = 0 cm of dots2, minimum width=2.7cm] (sn) {a = 2\^{}32 - 1};

    \node [pc, right = 1.5 cm of s65534, minimum width=4.3cm]
    (s65534_icmp){a = 65534; b = 0};
    \node [pc, right = 1.5 cm of s65535, minimum width=4.3cm]
    (s65535_icmp){a = 65535; b = 1};

    \node [dots, above = 0.0 cm of s65534_icmp] (dots1_icmp){\LARGE$\vdots$};
    \node [dots, below = -0.2 cm of s65535_icmp] (dots2_icmp){\LARGE$\vdots$};

    \node [pc, right = 1.5 cm of s0, minimum width=4.3cm] (s0_icmp)
    {a = 0; b = 0};
    \node [pc, right = 1.5 cm of sn, minimum width=4.3cm] (sn_icmp)
    {a = 2\^{}32 - 1; b = 1};

    \node [empty, left  = 1 cm of s]  (start) {};
    \node [empty, right = 1 cm of s0_icmp] (s0end) {};
    \node [empty, right = 1 cm of s65534_icmp] (s65534end) {};
    \node [empty, right = 1 cm of s65535_icmp] (s65535end) {};
    \node [empty, right = 1 cm of sn_icmp] (snend) {};

    \draw [flow] (s.east) -- (s0.west) node [near end, above=1pt, sloped] {\texttt{call}} ;
    \draw [flow] (s.east) -- (s65534.west) node [near end, above=1pt, sloped] {\texttt{call}} ;;
    \draw [flow] (s.east) -- (s65535.west) node [near end, below=1pt, sloped] {\texttt{call}} ;
    \draw [flow] (s.east) -- (sn.west) node [near end, below=1pt, sloped] {\texttt{call}} ;

    \draw [flow] (s0) -- (s0_icmp) node [midway, above=0pt] {\texttt{icmp}};
    \draw [flow] (s65534) -- (s65534_icmp) node [midway, above=0pt] {\texttt{icmp}};
    \draw [flow] (s65535) -- (s65535_icmp) node [midway, above=0pt] {\texttt{icmp}};
    \draw [flow] (sn) -- (sn_icmp) node [midway, above=0pt] {\texttt{icmp}};

    \draw [flow, dashed] (s0_icmp.east) -- (s0end) node [empty, midway, above=2pt] {};
    \draw [flow, dashed] (s65534_icmp.east) -- (s65534end) node [empty, midway, above=2pt] {};
    \draw [flow, dashed] (s65535_icmp.east) -- (s65535end) node [empty, midway, above=2pt] {};
    \draw [flow, dashed] (sn_icmp.east) -- (snend) node [empty, midway, above=2pt] {};

    \draw [flow, dashed] (start) -- (s);
\end{tikzpicture}
}
\end{centering}

\bigskip

\noindent
On the other hand, in \SymDIVINE the `input` call generates single multi state,
where `a` represents a set of all possible 32-bit values. State space is then
branched when a comparison instruction `icmp` is executed. In this case, two
possible scenarios may happen. First, when `a` is smaller then 65535, a multi
state with `b = 0`, as result of comparison, is generated. Similarly for second case, when
the condition is satisfied, a state with appropriately constrained `a` is emmited.

\bigskip

\begin{centering}
\resizebox{\textwidth}{!}{
\begin{tikzpicture}[]
    \tikzstyle{every node}=[align=center, minimum width=1.25cm, minimum height=0.6cm]
    \tikzset{empty/.style = {minimum width=0cm,minimum height=1cm}}
    \tikzset{dots/.style = {draw=none}}
    \tikzset{>=latex}

    \node [pc, text width = 1cm] (s_sym) {$\bot$};
    \node [pc, text width = 3.7cm, right = 1.5 cm of s_sym, minimum width=2cm] (s_nd_sym)
    {a = \{0,\dots,2\^{}32 - 1\}};

    \node [empty, right = 4cm of s_nd_sym] (mid_sym) {};

    \node [pc, text width = 5cm, above = -0.45 cm of mid_sym, minimum width=5cm] (s1_sym)
    {a = \{0,\dots,65534\}\\b = \{0\}};
    \node [pc, text width = 5cm, below = -0.45 cm of mid_sym, minimum width=5cm] (s2_sym)
    {a = \{65535,\dots,2\^{}32 - 1\}\\b = \{1\}};

    \node [empty, left  = 1 cm of s_sym]  (start_sym) {};
    \node [empty, right = 1 cm of s1_sym] (s1end_sym) {};
    \node [empty, right = 1 cm of s2_sym] (s2end_sym) {};

    \draw [flow] (s_sym.east) -- (s_nd_sym.west) node [midway, above=0pt] {\texttt{call}};

    \draw [flow] (s_nd_sym.east) -- (s1_sym.west) node [midway, above=1pt,sloped] {\texttt{icmp}};
    \draw [flow] (s_nd_sym.east) -- (s2_sym.west) node [midway, below=1pt, sloped] {\texttt{icmp}};

    \draw [flow, dashed] (start_sym) -- (s_sym);
    \draw [flow, dashed] (s1_sym) -- (s1end_sym);
    \draw [flow, dashed] (s2_sym) -- (s2end_sym);
\end{tikzpicture}
}
\end{centering}

\end{example}

uring state space exploration \SymDIVINE employs a \SMT solver for two tasks.
irstly, in compare to \DIVINE, a detection of already visited state is much
more complicated. Comparison of two multi states is done by comparison of their
explicit part (i.e.~whether they are in same control location) and symbolic
part. Equality of symbolic part is achieved by subset comparison, i.e. the
symbolic parts are equal when one is subset of another and vice versa.
This subset comparison can be efficiently coded into \SMT formula and given to
\SMT solver.

The second use case of \SMT solver is when exploration algorithm needs to
determine whether a given state is reachable. This may be done by simple query
on satisfiability of path condition. For example if symbolic part of state is
represented by following formula: $x > 0 \wedge x < 0$, state is unreachable,
because formula is unsatisfiable. As a consequence \SymDIVINE does
not generate a given unreachable successor.

Talking about \SymDIVINE overall architecture, it mimics design of \DIVINE.
Generally a verification workflow consists of preprocessing part, where
generation of \LLVM bitcode and minor optimizations are done. In compare to
\DIVINE, symbolic approach does not assimilate \DIOS layer. Hence all scheduling
and memory management is done solely by verification core of \SymDIVINE. Except
that, we may distinguish 3 parts in verification core, an interpreter, a state
generator and exploration manager. They behave similarly as their \DIVINE
counterparts. In extension state space generator communicates with external
\SMT solver as described before.

%\begin{figure}[!ht]
%\centering
%\resizebox{\textwidth}{!}{
%\begin{tikzpicture}[>=stealth',shorten >=1pt,auto,node distance=4em, <->]
%\tikzset{>=latex}
%
%    \tikzstyle{smt}=[fill=ucla!40]
%    \node [component](clang) {clang -emit-llvm};
%    \node [clabel, above = 0.3 cm of clang] (preprocessing) {Preprocessing};
%    \node [component, below = 0.5 cm of clang](lart) {LART};
%
%    \node [component, right = 0.6 cm of lart, ](interpreter) {\LLVM interpreter};
%    \node [component, right = 0.5 cm of interpreter, minimum width=1 cm](generator) {State space generator};
%
%    \node [component, smt, right = 0.5 cm of generator, minimum width=1 cm, text width=1 cm](smt){\SMT solver};
%    \node [component, above = 0.5 cm of generator, minimum width=1 cm](exploration) {Exploration algorithm};
%    \node [clabel, right = 0.5 cm of preprocessing] (symdivine) {SymDIVINE};
%
%    \node [right = 0.5 cm of exploration ] (res)
%    {\color{apple}{Valid}\color{pruss}/\color{orioles}{Error}};
%
%    \begin{pgfonlayer}{background}
%        \node[runtime, outer, fit = (clang) (lart) (preprocessing)] (prepbox) {};
%    \end{pgfonlayer}
%
%    \begin{pgfonlayer}{background}
%        \node[verification, outer, fit = (interpreter) (generator) (exploration) (symdivine) ] (preprocessing) {};
%    \end{pgfonlayer}
%
%    \node [left = 1.5 cm of clang, color=pruss] (start) {\Cpp{} program};
%    \node [right = 2 cm of exploration] (end) {};
%    \node [below = 2.3 cm of start, color=pruss] (property) {property};
%
%    \draw [flow] (clang) -- (lart);
%
%    \draw [flow] (lart) -- (interpreter);
%
%    \draw [flow, <->] (interpreter) -- (generator);
%
%    \draw [flow, <->] (smt) -- (generator);
%
%    \draw [flow] (generator) -- (exploration);
%
%    \draw [flow, dashed] (start) -- (clang);
%    \draw [flow, dashed] (property) -| (generator);
%    \draw [flow, dashed] (exploration) -- (res);
%\end{tikzpicture}
%}
%\caption{ \SymDIVINE{}'s verification workflow similar to \DIVINE{}'s. It consists
%of two steps, a preprocessing part where a suitable \LLVM bitcode is created. In compare to
%\DIVINE, compilation is done by unmodified compiler and \LART transformations
%are used only slightly. We would like to note, that only verified program
%attends a compilation process, since no \SymDIVINE proprietary runtime (as \DIOS) is
%needed. On the other hand \SymDIVINE needs to simulate scheduling directl in
%interpreter and state space generator. In compare to \DIVINE a verification part
%of \SymDIVINE maintains similar architercture. The only big difference is
%interfce to external \SMT solver.}\label{fig:symdivine}
%\end{figure}

Even though \SymDIVINE may sound as suitable instrument for verification of open
programs, it comes with couple of issues.

\add{ symdivine nedostatky }
