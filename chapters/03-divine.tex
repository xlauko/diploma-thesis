\chapter{DIVINE and its friends}\label{ch:divine}

In this chapter, we will look under the hood of the \DIVINE model
checker~\cite{Divine17}.  On top of that, we will try to tackle the problem of
handling inputs by using symbolic model checking as introduced by Barnat et
al.~in~\cite{Barnat14}. We will emphasize the differences between the
explicit-state model checking algorithm and the symbolic model checking algorithm.
The differences will be crucial in the next chapter where we will transform the
program given to \DIVINE in such a way, that it will simulate the symbolic approach.
Hence \DIVINE will be able to verify programs with inputs.

For the purpose of the transformation, only minimal knowledge of the architecture of
\DIVINE is needed. Our interest is mostly in the \LLVM interpreter (\DIVM) and
the exploration algorithm, since only they interacts with a transformed program.
Hence we will omit broader technicalities. Deeper description of \DIVINE can be
found in a tool paper of latest release\sidenote{It is \DIVINE 4.0 at the time
of writing this thesis.} \cite{Divine17} or on the project website
\href{https://divine.fi.muni.cz/}{divine.fi.muni.cz}.

At the end of the chapter we shortly summarize approaches of other verification
tools, that use symbolic representation of data, and compare them to our approach.

\section{Model checking with \DIVINE}

\DIVINE is a
modular platform for verification of real world programs. The overall
architecture
can be divided into 2 parts:
\begin{itemize}
    \item a~verification core that provides tools for \LLVM interpretation and
        state space exploration,
    \item a runtime environment, whose purpose is to give support of
        language features like memory allocation, threads and standard
        libraries.
\end{itemize}
Both parts are further split into several components,
with precisely defined interfaces between them, see \autoref{fig:architecture}.

The runtime environment consists of user's program accompanied with necessary
libraries. For user's program \DIVINE is distributed with a \Cpp{} standard
library and \textsc{posix} threading library suited for \DIVINE interpreter. To
simulate the real world conditions, user's program is executed on top of \DIVINE
operating system (\DIOS). Besides error handling, the main purpose of \DIOS is
to manage threads and their scheduling. The communication between program and
the \DIOS is done through a set of `syscalls` that are implemented by \DIOS.

The verification core consists of 2 main components: a \DIVINE virtual machine
(\DIVM) that interprets \LLVM bitcode and provides a state space generator, and
verification algorithm that defines how the generated state space is
explored. For communication with \DIVM we will distinguish \code{hypercalls},
further described in next section.

\begin{figure}[!ht]

\resizebox{\textwidth}{!}{

\begin{tikzpicture}[>=stealth,shorten >=1pt,auto,node distance=4em, <->]

\tikzstyle{bcomponent} = [
     color=pruss,
    fill=white,
    thick,
    draw,
    text centered,
    minimum height= 1cm,
    minimum width=2.2 cm,
    text width=6 cm,
];

\node [bcomponent] (input) {User's program $+$ libraries};
\node [clabel, above = 0cm of input] (renv) {Runtime environment};
\node [bcomponent, below = 0cm of input] (runtime) {\Cpp{} standard libraries, \texttt{pthreads}};
\node [fnlabel, right = 1.7cm of runtime] (syslabel) {syscalls};
\node [color = pruss, left = 2cm of runtime] (divine) {\large\DIVINE};
\node [bcomponent, below = 0cm of runtime] (dios) {\DIOS};

\node [clabel, below = 0.4cm of dios] (venv) {Verification core};
\node [fnlabel, left = 3.2cm of venv] (hyplabel) {hypercalls};
\node [bcomponent, below = 0cm of venv] (divm) {\DIVM};
\node [bcomponent, below = 0cm of divm] (vc) {Verification algorithm};


\begin{pgfonlayer}{background}
     \node[runtime, outer, fit = (renv) (input) (runtime) (dios)] (runtimebox) {};
\end{pgfonlayer}

\begin{pgfonlayer}{background}
  \node[verification, outer, fit = (venv) (divm) (vc)] (verificationbox) {};
\end{pgfonlayer}

\draw [-,dashed, very thick, color = pruss] ([xshift=4cm]input.south east) -- ([xshift=-4cm]input.south west);
\draw [flow,rectangle connector=1.5cm] (input.east) to (dios.east);
\draw [flow,rectangle connector=0.75cm] (runtime.east) to (dios.east);

\draw [flow,rectangle connector=-1.5cm] (runtime.west) to (divm.west);
\draw [flow,rectangle connector=-0.75cm] (dios.west) to (divm.west);
\end{tikzpicture}
}

\caption{Layout of \DIVINE architecture.}\label{fig:architecture}
\end{figure}

\section{\DIVINE virtual machine}

The \DIVINE virtual machine aims to provide a bare minimum for \LLVM-based model
checking. This involves execution of instructions, memory management and support
for nondeterministic choice\sidenote{Nondeterministic choice
is mainly used for scheduling of threads.} \cite{RockaiCB17}.

Besides interpretation of \LLVM bitcode, \DIVM stores a representation of a
program state. The verification algorithm may ask \DIVM for successors of a
given state. Afterwards \DIVM interprets \LLVM bitcode from that state until it
reaches some interrupt.\sidenote{Interpreted \LLVM instructions then represent
an edge between two states.} We will call this process edge generation.  During
the interpretation \DIVM checks for potential property violation.

A program state is represented by memory configuration encoded as a graph.
Nodes of the graph represent objects (e.g.~results of allocation, global
variables) and edges represent pointers between these objects.

For precise verification, the runtime environment is equipped with scheduler,
that defines an interleaving of threads. In order to simulate all interleavings,
\DIVM delivers a hypercall for nondeterministic choice. This hypercall can be
used from the runtime to branch the state space (i.e.~simulate all possible
thread interleavings). Hence it will be useful also/
for us to simulate a nondeterminism in abstracted programs.

\begin{example}
A hypercall for nondeterministic choice in \DIVM is called \code{\_\_vm\_choose}.
It gives an ability to branch the state space by enumerating all possible values
from a given range.

\begin{minted}[frame=none]{cpp}
int n  = __vm_choose( range );
\end{minted}
\noindent
In a code above, a nondeterministic choice creates a successor state for each
\texttt{n} from a given \texttt{range}.

\end{example}
\subsection{State space reductions}

In order to enable verification of real world programs, \DIVINE needs to employ
some state space reduction technique. For this purpose a so called
$\tau$-reduction and heap symmetry-based reduction are used \cite{Rockai13,
RockaiCB17}. \marginpar{$\tau$-reduction eliminates superfluous intermediate
states and leave only those that are interesting for the verification.} It is
demanded that state space reductions preserve all temporal properties
verified by \DIVINE.

In \LLVM, a large subset of instructions have no observable effect for other
threads. This holds for instructions that manipulate registers (registers are
private for the executed function) and those instructions that manipulate only
private memory of a thread. Hence \DIVINE does not need to emit a new state on
every interpreted instruction, but only when an observable action is reached. In
the resulting state space, edges correspond to sequences of non-observable
instructions.

In \DIVM, a support for the $\tau$-reduction is implemented by
\code{interrupt\_mem} and \code{interrupt\_cfl} hypercalls. These calls enable
signalization of an observable action to the state space generator and may cause
an interrupt in the edge generation. The \code{interrupt\_mem} hypercall
signals to \DIVM that a memory operation was executed. For signalization of
potential a loop in the program state space, the \code{interrupt\_cfl} hypercall
is used.

In the transformation we will use these hypercalls to notify the interpreter
that potential loop in the abstract state space may have occurred (i.e.~we need
to insert them to the transformed program).

\subsection{Verification workflow}

Verification in \DIVINE is split into two phases, see
\autoref{fig:verification}.  A preprocessing phase, where an input program is
transformed into a form suitable for \DIVM. In the second phase, the
transformed program is processed by \DIVM and a verification algorithm.

Since most of this thesis extends the preprocessing part, let us have a closer
look on the program transformations. The transformations are similar to \LLVM
optimization passes (they work in an \LLVM -to-\LLVM manner). They modify an
input program in order to extend the capabilities of the model checker. For
example, verification of programs with weak memory models is done via a
transformation \cite{Still16}, verification of programs with exceptions
\cite{Still17} and minor optimizations like interrupt insertion for faster
scheduling.

The \LLVM bitcode is preprocessed by a stand-alone tool named
\LART~\cite{Rockai15}.\sidenote{\emph{\LLVM} \emph{abstraction \& refinement
tool}} The main motivation behind \LART is to provide preprocessing for \LLVM
-based verification tools, simplifying their job by reducing the problem size
without compromising soundness of verification.

An abstraction tool, \LART was never fully implemented and till this thesis it
was meant only as a proof of concept. The main aim of this thesis is to provide
a core analysis for \LART to be able to inject arbitrary abstraction into a
program.

\begin{figure}[!ht]

\resizebox{\textwidth}{!}{
\begin{tikzpicture}[>=stealth',shorten >=1pt,auto,node distance=4em, <->]
\tikzset{>=latex}

    \tikzstyle{smt}=[fill=ucla!40]
    \node [component](cc) {\DIVINE compiler};
    \node [clabel, above = 0.7 cm of cc] (preprocessing) {Preprocessing};
    \node [component, right = 0.5 cm of cc](lart) {\LART};
    \node [emptycomponent, dashed, thick, below = 0.6 cm of cc] (dios) {DiOS and libraries};

    \node [component, right = 0.6 cm of lart, ](interpreter) {Interpreter};
    \node [component, right = 0.5 cm of interpreter, minimum width=1 cm](generator) {State space generator};

    \node [component, below = 0.6 cm of generator, minimum width=1 cm](exploration) {Exploration algorithm};
    \node [clabel, right = 3 cm of preprocessing] (divine) {DIVINE};

    \node [right = 0.5 cm of exploration ] (res)
    {\color{apple}{Valid}\color{pruss}/\color{orioles}{Error}};

    \node [clabel, above = 0.9 cm of interpreter.west, anchor = west] (divml)
    {DiVM};
    \node[emptycomponent, dashed, fit = (interpreter) (generator) (divml)] (divm) {};

    \begin{pgfonlayer}{background}
        \node[runtime, outer, fit = (cc) (lart) (preprocessing)(dios)] (prepbox) {};
    \end{pgfonlayer}

    \begin{pgfonlayer}{background}
        \node[verification, outer, fit = (interpreter) (generator) (exploration) (divine) (divm) ] (di) {};
    \end{pgfonlayer}

    \node [left = 1.5 cm of cc, color=pruss] (start) {\Cpp{} program};
    \node [right = 2 cm of exploration] (end) {};
    \node [below = 2.3 cm of start, color=pruss, text width = 1.5 cm] (property) {\centering property and\\ options};

    \draw [flow] (cc) -- (lart);

    \draw [flow] (dios) -- (cc);

    \draw [flow] (lart) -- (interpreter);

    \draw [flow, <->] (interpreter) -- (generator);

    \draw [flow] (generator) -- (exploration);

    \draw [flow, dashed] (start) -- (clang);
    \draw [flow, dashed] (property) -| (interpreter);
    \draw [flow, dashed] (property) -| (lart);
    \draw [flow, dashed] (exploration) -- (res);
\end{tikzpicture}
}
\caption{Verification process in \DIVINE consists of preprocessing and
state space exploration. In preprocessing step an input code is
compiled and linked with \DIOS runtime and \DIVINE support
libraries. Then the bitcode is instrumented by \LART producing a suitable \LLVM bitcode with \DIVM hypercalls. And finally the bitcode is interpreted in \DIVM and the verification result is produced.}\label{fig:verification}
\end{figure}

\section{Symbolic model checking with \SymDIVINE}\label{sub:symdivine}

To extend \DIVINE abilities a prototype tool with capability of handling inputs
was designed -- \SymDIVINE.  This tool is proof-of-concept that is based on the
idea of \emph{control-explicit data-symbolic} model checking \cite{Barnat14}.
Since this was only prototype project, we would like to incorporate its
abilities into the fully powered \DIVINE. In following text we will describe
differences between both tools, in order to recognize which parts need to be added
to \DIVINE verification toolchain. Especially we want to detect, what are we
able to simulate by transformed program directly and what parts of \SymDIVINE
needs to be incorporated into \DIVINE.

\subsection{Control-explicit data-symbolic model checking}

In comparison to exhaustive enumeration of states in \DIVINE, a control-explicit
data-symbolic approach tries to group states into sets, when they differ only in
data values but not in control location. In this way \SymDIVINE is able to
represent inputs as sets of possible values. These sets, also called \emph{multi
states}, are described by an explicit control location and symbolic
representation of data (more precise description will be in
\autoref{sec:multistates}). In \SymDIVINE, exploration algorithm works directly
with multi states. In order to explore a state space, \SymDIVINE needs to be
able to distinguish whether two multi states represent the same set of explicit
states. Sadly this operation can be quite time-consuming. On the other hand
symbolic approach may bring exponential memory savings and avoid state space
explosion caused by inputs.

\subsection{Representation of multi states} \label{sec:multistates}

A challenging part of symbolic model checking is how to define a suitable
representation of states. Representation of control location (explicit part of
multi state) is straightforward. \SymDIVINE just needs to store
a program location for each thread. On the other hand coming up with a good
representation of symbolic data may be quite challenging. Since \SymDIVINE aims
for bit-precise verification a suitable choice is representation of data
by fixed sized bit-vector formulae~\cite{Bauch14}.

The theory over fixed sized bit-vectors is a many-sorted first-order theory with
infinitely many sorts $\sort{n}$ corresponding to bit-vectors of length $n$. The
only predicate symbols in the BV theory are $=$, $\leq_u$, and $\leq_s$,
representing equality, unsigned inequality of binary-encoded natural numbers,
and signed inequality of integers in $2$'s complement representation,
respectively. Functions symbols in the theory are $+, \times, \div, \&, \mid,
\oplus, \ll, \gg, \cdot, \extract{n}{p}$, representing addition, multiplication,
unsigned division, bit-wise and, bit-wise or, bit-wise exclusive or, left-shift,
right-shift, concatenation, and extraction of $n$ bits starting from position
$p$, respectively. For a precise description of the syntax and semantics of the
bit-vector theory, we refer the reader to Hadarean \cite{Hadarean14}.

Since \SymDIVINE does not support dynamic memory allocation, representation
of symbolic data becomes much simpler. In representation via bit-vector
\SMT formulae, an unambiguous identification of symbolic variables is crucial.
Therefore \SymDIVINE uses names given by function
segment on stack and offset of given variable in the corresponding segment \cite{Mrazek16}.
Moreover, manipulations with symbolic data are represented by corresponding functional symbols.

Besides remembering current values of variables, \SymDIVINE has to keep for
some variables previous evaluations (also called generations). This happens when
a value of symbolic variable depends on another symbolic variable.

\begin{example}\label{ex:gen}
Having following C code \SymDIVINE has to store multiple generations of variable
\code{a}, because after the evaluation of all statements \SymDIVINE needs to know
that, \code{b} is equal to \code{a} from first line (first generation of
\code{a}) and \code{a} (in second generation) is equal to $0$.

\begin{minted}{cpp}
int a = 10;
int b = a;
a = 0;
\end{minted}

\end{example}

A symbolic representation of data is further structured into two parts -- a
program \emph{path condition} and data \emph{definitions}. The path condition is a
conjunction of formulae that represents a restriction of the data that have been
collected during branching along the path leading to the current location.
Definitions, on the other hand, are made of a set of formulae in the form
\emph{variable = expression} that describe internal relations among variables.
Definitions are produced as a result of an assignment or arithmetic
instructions. The structure of symbolic data representation allows for a
precise description of what is needed for model checking, but it lacks a
canonical representation. As a matter of fact, the equality of multi states
cannot be performed as a syntax equality, instead, \SymDIVINE employs an \SMT
solver and quantified formulae to check the satisfiability of a path condition
and to decide the equality of two multi states \cite{Mrazek16}.

\subsection{State space exploration}

When program reads some input, an explicit state model checker emits a
new successor for every possible input value. It is important to remark that all of
these states differ only in a single value, but they occupy same control flow
location. Hence a same sequence of instructions is applied to them.

\marginpar{Nondeterministic input values for \SymDIVINE are introduced by
\code{\_\_VERIFIER\_nondet\_\{type\}}, where \code{\{type\}} defines type of
input value.}

On the other hand a symbolic approach emits only one multi state. We may look at
multi state as it represents a set of purely explicit states (defined by
set-based reducion~\cite{Havel14}). See an example \ref{ex:reduction} for comparison of
explicit state space and state space reduced by set-based reduction.

The only event that splits a state space of multi states is a branch in the
program control flow, that depends on a symbolic value (conditional \code{br}
instruction in \LLVM). As a consequence a pair of multi states has to be
generated, one that corresponds to state when the branching condition was true and
second when the condition was false. Additionally, according to comparison result
corresponding values in multi states have to be constrained based on comparison
condition. For better explanation see following example \ref{ex:reduction}.

\begin{example}\label{ex:reduction}
Let's have a look on the impact of set-based reduction on program with input and
branching. The corresponding \LLVM bitcode is on the right side.

\begin{center}
\begin{minipage}[t]{.47\textwidth}
\begin{minted}{cpp}
unsigned int i = input();
if (i >= 65535) {
    ...
} else {
    ...
\end{minted}
\end{minipage}\hfill
\begin{minipage}[t]{.47\textwidth}
\begin{minted}{llvm}
%i = call i32 @input()
%b = icmp uge i32 %i, 65535
%br i1 %b, label %t, label %e
\end{minted}
\end{minipage}
\end{center}

\bigskip
\noindent
In order to precisely verify a program with input, \DIVINE has to generate all
possible evaluations of the \code{input} call. Hence the resulting state space
suffers from the great state space explosion. We may notice that the state
space is branched when the \code{input} call is executed, but following
execution continues in a sequential manner:

\begin{centering}
\resizebox{\textwidth}{!}{
\begin{tikzpicture}[]
    \tikzstyle{every node}=[align=center, minimum width=1.25cm, minimum height=0.6cm]
    \tikzset{empty/.style = {minimum width=0cm,minimum height=1cm}}
    \tikzset{dots/.style = {draw=none}}
    \tikzset{>=latex}
    \node [pc, text width = 1 cm] (s) {$\bot$};
    \node [right = 3cm of s] (mid) {};

    \node [pc, above = -0.25 cm of mid, minimum width=2.7cm] (s65534){i = 65534};
    \node [pc, below = -0.25 cm of mid, minimum width=2.7cm] (s65535){i = 65535};

    \node [dots, above = 0 cm of s65534] (dots1){\LARGE$\vdots$};
    \node [dots, below = -0.2 cm of s65535] (dots2){\LARGE$\vdots$};

    \node [pc, above = -0.2 cm of dots1, minimum width=2.7cm] (s0) {i = 0};
    \node [pc, below = 0 cm of dots2, minimum width=2.7cm] (sn) {i = 2\^{}32 - 1};

    \node [pc, right = 1.5 cm of s65534, minimum width=4.3cm]
    (s65534_icmp){i = 65534; b = 0};
    \node [pc, right = 1.5 cm of s65535, minimum width=4.3cm]
    (s65535_icmp){i = 65535; b = 1};

    \node [dots, above = 0.0 cm of s65534_icmp] (dots1_icmp){\LARGE$\vdots$};
    \node [dots, below = -0.2 cm of s65535_icmp] (dots2_icmp){\LARGE$\vdots$};

    \node [pc, right = 1.5 cm of s0, minimum width=4.3cm] (s0_icmp)
    {i = 0; b = 0};
    \node [pc, right = 1.5 cm of sn, minimum width=4.3cm] (sn_icmp)
    {i = 2\^{}32 - 1; b = 1};

    \node [empty, left  = 1 cm of s]  (start) {};
    \node [empty, right = 1 cm of s0_icmp] (s0end) {};
    \node [empty, right = 1 cm of s65534_icmp] (s65534end) {};
    \node [empty, right = 1 cm of s65535_icmp] (s65535end) {};
    \node [empty, right = 1 cm of sn_icmp] (snend) {};

    \draw [flow] (s.east) -- (s0.west) node [near end, above=1pt, sloped] {\texttt{call}} ;
    \draw [flow] (s.east) -- (s65534.west) node [near end, above=1pt, sloped] {\texttt{call}} ;;
    \draw [flow] (s.east) -- (s65535.west) node [near end, below=1pt, sloped] {\texttt{call}} ;
    \draw [flow] (s.east) -- (sn.west) node [near end, below=1pt, sloped] {\texttt{call}} ;

    \draw [flow] (s0) -- (s0_icmp) node [midway, above=0pt] {\texttt{icmp}};
    \draw [flow] (s65534) -- (s65534_icmp) node [midway, above=0pt] {\texttt{icmp}};
    \draw [flow] (s65535) -- (s65535_icmp) node [midway, above=0pt] {\texttt{icmp}};
    \draw [flow] (sn) -- (sn_icmp) node [midway, above=0pt] {\texttt{icmp}};

    \draw [flow, dashed] (s0_icmp.east) -- (s0end) node [empty, midway, above=2pt] {};
    \draw [flow, dashed] (s65534_icmp.east) -- (s65534end) node [empty, midway, above=2pt] {};
    \draw [flow, dashed] (s65535_icmp.east) -- (s65535end) node [empty, midway, above=2pt] {};
    \draw [flow, dashed] (sn_icmp.east) -- (snend) node [empty, midway, above=2pt] {};

    \draw [flow, dashed] (start) -- (s);
\end{tikzpicture}
}
\end{centering}

\bigskip

\noindent
On the other hand, in \SymDIVINE the \code{input} call generates a single multi
state, where '\code{i}' represents a set of all possible 32-bit values. State
space is then branched when a comparison instruction \code{icmp} is executed. In
this case, two possible scenarios may happen. First, when '\code{i}' is smaller
than 65535, a multi state with \code{b = 0}, as a result of comparison, is
generated. Similarly for the second case, when the condition is satisfied, a
state with appropriately constrained '\code{i}' is generated:

\bigskip

\begin{centering}
\resizebox{\textwidth}{!}{
\begin{tikzpicture}[]
    \tikzstyle{every node}=[align=center, minimum width=1.25cm, minimum height=0.6cm]
    \tikzset{empty/.style = {minimum width=0cm,minimum height=1cm}}
    \tikzset{dots/.style = {draw=none}}
    \tikzset{>=latex}

    \node [pc, text width = 1cm] (s_sym) {$\bot$};
    \node [pc, text width = 3.7cm, right = 1.5 cm of s_sym, minimum width=2cm] (s_nd_sym)
    {i = \{0,\dots,2\^{}32 - 1\}};

    \node [empty, right = 4cm of s_nd_sym] (mid_sym) {};

    \node [pc, text width = 5cm, above = -0.45 cm of mid_sym, minimum width=5cm] (s1_sym)
    {i = \{0,\dots,65534\}\\b = \{0\}};
    \node [pc, text width = 5cm, below = -0.45 cm of mid_sym, minimum width=5cm] (s2_sym)
    {i = \{65535,\dots,2\^{}32 - 1\}\\b = \{1\}};

    \node [empty, left  = 1 cm of s_sym]  (start_sym) {};
    \node [empty, right = 1 cm of s1_sym] (s1end_sym) {};
    \node [empty, right = 1 cm of s2_sym] (s2end_sym) {};

    \draw [flow] (s_sym.east) -- (s_nd_sym.west) node [midway, above=0pt] {\texttt{call}};

    \draw [flow] (s_nd_sym.east) -- (s1_sym.west) node [midway, above=1pt,sloped] {\texttt{icmp}};
    \draw [flow] (s_nd_sym.east) -- (s2_sym.west) node [midway, below=1pt, sloped] {\texttt{icmp}};

    \draw [flow, dashed] (start_sym) -- (s_sym);
    \draw [flow, dashed] (s1_sym) -- (s1end_sym);
    \draw [flow, dashed] (s2_sym) -- (s2end_sym);
\end{tikzpicture}
}
\end{centering}

\end{example}

During state space exploration, \SymDIVINE employs an \SMT solver for two tasks.
First, in contrast to \DIVINE, detection of an already visited state is much
more complicated. Comparison of two multi states is done by comparison of their
explicit part (i.e.~whether they are in same control location) and symbolic
part. Equality of symbolic parts is checked by subset comparison, i.e. the
symbolic parts are equal when the sets of explicit states are mutuall subsets.
This subset comparison is encoded into an \SMT formula and given to an \SMT solver.

The second use case of the \SMT solver is when \SymDIVINE needs to determine
whether a given state is reachable. This may be done by a simple query on the
satisfiability of a path condition.\sidenote{Consider a path condition: $x > 0
\wedge x < 0$, which is unsatisfiable because there is no such $x$ that is
smaller and larger than zero at once. As a consequence, \SymDIVINE does not
generate an unreachable successor.} The path condition is a formula that
represents visited constraints in the execution taken so far (i.e.  when
\SymDIVINE takes conditional branching, for example on condition $x < 10$, it
stores this constrain to the path condition).

\subsection{Multi state equality check}\label{subsec:equality}

To recognize that two multi states are equal, \SymDIVINE checks by \SMT solver
whether the symbolic parts of two multi states represent the same set of values.
Formally we will describe a multi state as a triplet $(c, m, \varphi)$, where
$c$ is a control part, $m$ is a memory shape and $\varphi$ is a symbolic part of
the state.

Let $s_1$ and $s_2$ are multi states with same explicit part. That is, when $s_1
= (c, m, \varphi)$ and $s_2 = (c, m, \psi)$ for the control part $c$ and memory
shape $m$ and formulae $\varphi$ and $\psi$. Let $\free{\varphi} = \{x_1, \dots,
x_n\}$ and $\free{\psi} = \{ y_1, \dots, y_m\}$ denote free variables of
$\varphi$ and $\psi$. Moreover, we denote the set of program variables at the
control location $c$ as $\pVars{c}$. Then for each program variable $p$ exists a
variable $x^p$ in $\varphi$ that represents the last generation of the program
variable $p$ in the multi state $s_1$. Analogously, the last generation of the
program variable $p$ in $s_2$ is represented by a variable $y^p$ in $\psi$.

To decide whether the sets of concrete states represented by $s_1$ and $s_2$ are
equal, \SymDIVINE uses a formula $\notsubseteq(s_1, s_2)$, which is satisfiable
precisely if there is a state represented by $s_1$ that is not represented by
$s_2$:

\[
  \notsubseteq(s_1, s_2) \defeq \varphi~\land~ \forall y_1 \ldots y_m
  \, \Big ( \psi \Rightarrow \bigvee_{p \in \pVars{c}} (x^p \not =
  y^p) \Big)
\]

The equality of two multi states is then decided by using an \SMT solver: the
states $s_1$ and $s_2$ are equal precisely if both of formulas
$\notsubseteq(s_1, s_2)$ and $\notsubseteq(s_2, s_1)$ are unsatisfiable, i.e.
there is no memory valuation that is represented only by one of the
multi-states. However, the equality check requires a quantified \SMT query,
which is more expensive than a quantifier-free one.

\add{ add citation of Symdivine paper }

\subsection{Architecture of \SymDIVINE}
Talking about \SymDIVINE overall architecture, it mimics the design of \DIVINE.
Generally the verification workflow consists of a preprocessing part, where
\LLVM bitcode is generated and minor optimizations are done. In comparison to
\DIVINE, symbolic approach does not assimilate \DIOS layer. Hence all scheduling
and memory management is done solely by the verification core of \SymDIVINE. Except
that, we may distinguish 3 parts in the verification core, an interpreter, a state
generator and exploration manager. They behave similarly as their \DIVINE
counterparts. Except the \SymDIVINE's state space generator, that needs to
communicate with \SMT solver, to execute equality and reachability queries.

\begin{figure}[!ht]
\centering
\resizebox{\textwidth}{!}{
\begin{tikzpicture}[>=stealth',shorten >=1pt,auto,node distance=4em, <->]
\tikzset{>=latex}

    \tikzstyle{smt}=[fill=ucla!40]
    \node [component](clang) {clang -emit-llvm};
    \node [clabel, above = 0.3 cm of clang] (preprocessing) {Preprocessing};
    \node [component, below = 0.5 cm of clang](lart) {LART};

    \node [component, right = 0.6 cm of lart, ](interpreter) {\LLVM interpreter};
    \node [component, right = 0.5 cm of interpreter, minimum width=1 cm](generator) {State space generator};

    \node [component, smt, right = 0.5 cm of generator, minimum width=1 cm, text width=1 cm](smt){\SMT solver};
    \node [component, above = 0.5 cm of generator, minimum width=1 cm](exploration) {Exploration algorithm};
    \node [clabel, right = 0.5 cm of preprocessing] (symdivine) {SymDIVINE};

    \node [right = 0.5 cm of exploration ] (res)
    {\color{apple}{Valid}\color{pruss}/\color{orioles}{Error}};

    \begin{pgfonlayer}{background}
        \node[runtime, outer, fit = (clang) (lart) (preprocessing)] (prepbox) {};
    \end{pgfonlayer}

    \begin{pgfonlayer}{background}
        \node[verification, outer, fit = (interpreter) (generator) (exploration) (symdivine) ] (preprocessing) {};
    \end{pgfonlayer}

    \node [left = 1.5 cm of clang, color=pruss] (start) {\Cpp{} program};
    \node [right = 2 cm of exploration] (end) {};
    \node [below = 2.3 cm of start, color=pruss] (property) {property};

    \draw [flow] (clang) -- (lart);

    \draw [flow] (lart) -- (interpreter);

    \draw [flow, <->] (interpreter) -- (generator);

    \draw [flow, <->] (smt) -- (generator);

    \draw [flow] (generator) -- (exploration);

    \draw [flow, dashed] (start) -- (clang);
    \draw [flow, dashed] (property) -| (generator);
    \draw [flow, dashed] (exploration) -- (res);
\end{tikzpicture}
}
\caption{ Verification workflow of \SymDIVINE is similar to \DIVINE{}'s. It
    consists of two steps, a preprocessing part where a suitable \LLVM bitcode
    is created. In comparison to \DIVINE, compilation is done by unmodified
    compiler and \LART transformations are used only slightly. We would like to
    note, that only verified program attends a compilation process, since no
    \SymDIVINE proprietary runtime (as \DIOS) is needed. On the other hand
    \SymDIVINE needs to simulate scheduling directly in interpreter and state
    space generator. In comparison to \DIVINE a verification part of \SymDIVINE
    maintains similar architecture. The only big difference is an interface to
    external \SMT solver.}\label{fig:symdivine}
\end{figure}


\subsection{Drawbacks of \SymDIVINE}
Even though \SymDIVINE may sound as a suitable instrument for verification of
programs with inputs, it has a few issues. First of all, since
\SymDIVINE was designed only as a prototype tool, it does not provide full \Cpp{}
language support (for instance verification of program with exceptions).
Likewise, \SymDIVINE does not provide support for \Cpp{} standard library.
Moreover, it aims only for symbolic manipulation with integers, hence it
does not provide full support of the \LLVM instruction set, especially memory
manipulations and symbolic indices.

Another drawback of \SymDIVINE is that it can not interpret programs with
explicit memory manipulations, since \SymDIVINE interprets symbolically all the
instructions. Lastly, from the monolithic architecture of \SymDIVINE emerges a
disadvantage, that it is hardly expendable. Consequently it is hard to maintain
the core of the tool correct when adding the new features, what is
crucial for a verification tool.

We believe that including the \SymDIVINE approach into \DIVINE can get rid
of the most of these drawbacks. Especially it enables full support of \LLVM and
\Cpp{} exceptions. Moreover,explicit data can be handled explicitly by
well-tuned \DIVINE interpreter, hence support for explicit memory can be
maintained.

\begin{summary}
We have seen that \SymDIVINE is able to verify programs with inputs. In order to
do so, \SymDIVINE uses a bit-vector formulae to represent multi states ---
states that correspond to sets of explicit states. In comparison to
explicit-state model checking, the symbolic approach employs an \SMT solver for
equality check and reachability check.

In the transformation for \DIVINE we will need to reflect all those additions.
Concretely we will define the formula representation of a state and insert
the manipulations with it directly into the program. Similarly with path
condition.  Further, an interface to \SMT solver needs to be added to
\DIVINE's exploration algorithm in order to facilitate the \SMT queries.
\end{summary}

\section{State of the art tools}

A symbolic model checking is a well-known technique for a few
decades~\cite{McMillan93}. Besides \SymDIVINE there are other symbolic
model checkers such as NuSMV -- a tool based on the combination of \SMT and \BDD
technique \cite{Cimatti20}, or PRISM equipped with large variety of symbolic
algorithms \cite{Kwiatkowska20}. The big disadvantage of these tools is that
they do not accept \Cpp{} as a form of input. The only tool that we found that
is able to do a symbolic model checking on \Cpp{} programs is \textsc{redlib}
\cite{redlib}, which is no longer developed.

In compare to the approaches of symbolic model checkers mentioned earlier, we
want to design a transformation of a verified program in such a way, that the
program will simulate the symbolic machinery by itself. Hence it will need to
simulate manipulations with data definitions and similarly with the path
condition.

A similar approach to transformation was introduced by Khurshid et al.
\cite{Khurshid03} as an approach to symbolic execution. They have instrumented
Java code in such a manner that the resulting code behaves as its variables were
symbolic, this code is then processed by Java PathFinder \cite{Havelund20}. We
have designed a similar approach, an abstraction via program transformation for
\LLVM programs in the \autoref{ch:abstraction}.

\add{ TODO add limitations + comparison of approaches }


