\chapter{How to abstract a program?}\label{ch:appendixb}

\section{Archive and compilation}

The archive attached to this thesis contains sources of \DIVINE with
implementation of abstractions in \LART (located in
\texttt{divine\\/lart/abstract}).
Moreover, we include an entire set of benchmarks, with complete tables of results.

\subsection{Compilation}
To compile \DIVINE, unzip the distribution tarball and enter the
newly created directory
\begin{minted}{bash}
$ tar xvzf divine.tar.gz
$ cd divine
\end{minted}
The build is driven by a Makefile, and should be fully automatic. You only need to run:
\begin{minted}{bash}
$ make
\end{minted}
For further information, we kindly refer to the manual of the tool:
\url{https://divine.fi.muni.cz/manual.html}, which is also inclu\-ded with a tool
in \texttt{divine/doc} directory.

\section{Program abstraction}
In \DIVINE, we create an abstract variable using annotations or dedicated
functions. Provided capabilities are to denote a domain of scalar value in the
program. The abstract value does not need to be initialized since it is
inherently considered to have an arbitrary value from the given domain.

\subsection{Annotation}

To annotate variable we use a compiler attribute (\code{\_\_atribute\_\_}) with
a name of the domain. We predefine two types of annotations in
\code{abstract/domains.h} of symbolic and zero domain:
\begin{minted}{cpp}
#define _SYM __attribute__((__annotate__("lart.abstract.sym")))
#define _ZERO __attribute__((__annotate__("lart.abstract.zero")))
\end{minted}
In order to use annotations, user needs to include a header with annotation
definitions. Thereafter, annotations \code{\_SYM} and \code{\_ZERO} can be used in
the program:
\begin{minted}{cpp}
_SYM int x;
\end{minted}

To create a new domain, providing an implementation to the \LART, one can define
an annotation in a similar way. The name of the domain is parsed from the
annotation string.
\add{ add note about note functioning zero domain }

\subsection{\svcomp intrinsics}
Besides the annotations, \DIVINE supports \svcomp intrinsics as defined in
competition rules~\cite{svcomp}. Implementation of intrinsics using symbolic
domain is compiled and linked with a verified program and can be directly used
with verified program. \DIVINE provides following intrinsics:
\begin{itemize}
\item \code{\_\_VERIFIER\_nondet\_X()} to model nondeterministic values of
type \code{X} (\code{bool, char, int, uint, short, ushort, long, ulong},
\item \code{\_\_VERIFIER\_assume(int expression)} restrict a program \\behaviour
    according to the expression,
\item \code{\_\_VERIFIER\_assert(int condition)},
\item \code{\_\_VERIFIER\_error()},
\item \code{\_\_VERIFIER\_atomic\_begin()},
\item \code{\_\_VERIFIER\_atomic\_end()}.
\end{itemize}
The implementation of the intrinsics can be found in \DIVINE project subfolder:
\code{runtime/abstract/svcomp.cpp}.

\subsection{Run symbolic \DIVINE}

An annotated program can be verified by \DIVINE using symbolic algorithm:
\begin{minted}{bash}
./divine check --symbolic program.c
\end{minted}
A transformation is automatically executed before the verification starts. For
further parameters description, we kindly refer to a manual of the tool.

